%
% Copyright (c) 1998 University of Utah and the Flux Group.
% All rights reserved.
% 
% The University of Utah grants you the right to copy and reproduce this
% document or portions thereof for academic, research, evaluation, and
% personal use only, provided that (1) the title page appears prominently,
% and (2) these copyright and permission notices are retained in all copies.
% To arrange for alternate terms, contact the University of Utah at
% csl-dist@cs.utah.edu or +1-801-585-3271.
%

\section{X11 S3 Video Library}
\label{xvideo}

The x11video library contains a video driver supporting S3 video
cards.  This includes all video hardware supported by the XF86_S3
server (see \htmladdnormallink{http://www.xfree86.org}{http://www.xfree86.org}
for more information). 

\subsection{So how do I use this?}

The driver currently requires a configuration (XF86Config) file.  A
sample file is included in the x11/video directory.  You'll need to
make sure that it's readable as /etc/XF86Config.  This can either be
done through the BMOD (section~\ref{bmod-fs}) filesystem,
or by using the FreeBSD or Linux filesystem components.
	
This file is a subset of a standard XF86Config file and can only
contain the Monitor, Device and Screen sections.  The easiest way to
get things running is when you already have an XF86Config file for that
computer/monitor configuration.  In that case you can just strip out
everything from it but the sections listed above.  Otherwise you'll
need to make one from scratch, or from a sample config file from an
XFree86 distribution.

Currently the x11video driver only uses the default (first) entry in
the Screen section.  Unlike XFree86, there is no way to change or
specify color depths or resolution at runtime.

\subsection{Dependencies}

x11video depends on {\tt oskit_startup, oskit_unsupp, oskit_dev,
oskit_kern, oskit_freebsd_m, oskit_c}, and {\tt oskit_lmm}.

\subsection{API reference}

\api{s3_init_framebuffer}{Initializes the s3 video code}
\begin{apisyn}
	\cinclude{oskit/video/s3.h}

	\funcproto oskit_fb_t *s3_init_framebuffer(void);
\end{apisyn}

\api{s3_cmap_write}{Write a colormap entry}
\begin{apisyn}
	\cinclude{oskit/video/s3.h}
	
	\funcproto int s3_cmap_write(oskit_cmap_entry_t *c);
\end{apisyn}

\api{s3_cmap_read}{Read a colormap entry}
\begin{apisyn}
	\cinclude{oskit/video/s3.h}
	
	\funcproto int s3_cmap_read(oskit_cmap_entry_t *c);
\end{apisyn}

\api{s3_cmap_fg_index}{Return the colormap index for the foreground color}
\begin{apisyn}
	\cinclude{oskit/video/s3.h}
	
	\funcproto int s3_cmap_fg_index(void);
\end{apisyn}
