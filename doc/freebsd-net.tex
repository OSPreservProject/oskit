%
% Copyright (c) 1997-1999 University of Utah and the Flux Group.
% All rights reserved.
% 
% The University of Utah grants you the right to copy and reproduce this
% document or portions thereof for academic, research, evaluation, and
% personal use only, provided that (1) the title page appears prominently,
% and (2) these copyright and permission notices are retained in all copies.
% To arrange for alternate terms, contact the University of Utah at
% csl-dist@cs.utah.edu or +1-801-585-3271.
%
\long\def\lim#1{
	% I want to start on a new line. How do I do that???
	\ \\
		\textbf{Limitation/pecularity of the current implementation:} 
	\begin{quote} #1 \end{quote}}

\label{freebsd-net}

        \paragraph{A note of caution.} The design of the networking
	framework is nowhere near where it should be. However, we, the
	\oskit{} developers, felt that networking is important and
	decided to release what we have. 

	The current implementation has been tested with only the most 
	frequently used 
	calls for TCP and UDP over a single ethernet interface, 
	with a default router on the subnet to which the interface is connected.
	XXX We have done a simple router test with two interfaces.

\section{Introduction}
	This chapter describes the use and implementation of the 
	\freebsd{} TCP/IP networking stack. 

	\textbf{Note:}
	The file \texttt{startup/start_network.c} provides a
	convenience function \texttt{start_network()} to quickly
	start up your network.
	Then you may use the \texttt{socket} et al functions in
	the C library, as demonstrated in 
	\texttt{socket_bsd.c} in the example directory.
	The file \texttt{socket_com.c} contains
	the same example but uses the COM interfaces without
	C library support.
	
	\lim{
	We have not yet implemented ``principals'' as described in the 
	filesystem framework.  All operations run with full privileges.

	\texttt{oskit_socket_t} instances created by the networking
	stack do not currently implement the following 
	methods\footnote{An \texttt{OSKIT_E_NOTIMPL} error code will be
	returned.}:
	    \texttt{getsockopt},
	    \texttt{recvmsg/sendmsg}

	Also, the local loopback interface does not work because it is not
	properly set up. If you try to connect to \texttt{127.0.0.1} or
	to the local IP address, you'll see a division by zero trap in
	\texttt{freebsd/src/sys/netinet/ip_output.c:302} because the
	\texttt{if_mtu} field is uninitialized. Required fix is to call
	the required "ifconfig" functions for the loopback interface.
	%
	% Shouldn't be hard. this only to save you hours of debugging
	% if you see the div by zero trap.
	}

\apisec{Header Files}

\api{freebsd.h}{definitions for the \freebsd{}-derived networking code}
\begin{apisyn}
        \cinclude{oskit/net/freebsd.h}
\end{apisyn}

\begin{apidesc}
	Contains function definitions for the functions needed to
	initialize and use the \freebsd{} networking stack.

	It defines some convenience functions to initialize the code,
	to set up an interface and to establish a default route. 
	It defines \texttt{struct oskit_freebsd_net_ether_if} which
	is \textbf{opaque} to OS clients.
\end{apidesc}

%%%%%%%%%%%%%%%%%%%%%%%%%%%%%%%%%%%%%%%%%%%%%%%%%%%%%%%%%%%%%%%%%%%%%%%%%%%%%%

\apisec{Interfaces}

%%%%%%%%%%%%%%%%%%%%%%%%%%%%%%%%%%%%%%%%%%%%%%%%%%%%%%%%%%%%%%%%%%%%%%%%%%%%%%
%
% oskit_freebsd_net_init
%

\api{oskit_freebsd_net_init}{initialize the networking code}
\begin{apisyn}
        \cinclude{oskit/net/freebsd.h}

        \funcproto oskit_error_t 
		oskit_freebsd_net_init( 
			\outparam oskit_socket_factory_t **outfact); 
\end{apisyn} 
\ostonet

\begin{apidesc}
	This function initializes the \freebsd{} networking code.

\com{Apparently no longer true, because oskit_f_n_init calls oskit_f_init
     right off the bat.
	\lim{Must be called after the \freebsd{} dev library was initialized
	using \texttt{oskit_freebsd_init}}
}%com
	
\end{apidesc}

\begin{apiparm}
	\item[outfact] \emph{*outfact} will contain a 
	\texttt{oskit_socket_factory_t *} which can be used according to the
	specifications in the \oskit{} socket_factory COM interface in
	Chapter~\ref{net}.
\end{apiparm}

\begin{apiret}
        Returns 0 on success, or an error code specified in
        {\tt <oskit/error.h>}, on error.

	\lim{\texttt{oskit_freebsd_net_init} will always succeed, 
		and may panic otherwise.}
\end{apiret}

%%%%%%%%%%%%%%%%%%%%%%%%%%%%%%%%%%%%%%%%%%%%%%%%%%%%%%%%%%%%%%%%%%%%%%%%%%%%%%
%
%	oskit_freebsd_net_open_ether_if
%

\api{oskit_freebsd_net_open_ether_if}{find and open an ethernet interface}
\begin{apisyn}
        \cinclude{oskit/net/freebsd.h}

	\funcproto oskit_error_t
		oskit_freebsd_net_open_ether_if(
		struct oskit_etherdev *dev,
		\outparam struct oskit_freebsd_net_ether_if ** out_eif);
\end{apisyn}
\ostonet

\begin{apidesc}
	This function is a convenience function to open an ethernet
	device and create the necessary \texttt{oskit_netio_t} instances 
	to ``connect'' the \texttt{netio} device drivers to the 
	protocol stack. 

	Note: The code uses the following internal structure to keep track
	of an ethernet interface, defined in \texttt{oskit/net/freebsd.h}

\cstruct{oskit_freebsd_net_ether_if}{
    oskit_etherdev_t *dev;	/* ethernet device */
    oskit_netio_t   *send_nio;  /* netio for sending packets */
    oskit_netio_t   *recv_nio;  /* netio for receiving packets */
    oskit_devinfo_t     info;	/* device info */
    unsigned char   haddr[OSKIT_ETHERDEV_ADDR_SIZE]; /* MAC address */
    struct ifnet    *ifp;	/* actual interface seen by BSD code */
};

	\emph{ifp} is the actual interface as seen by the BSD code.
	\emph{recv_nio} points to the \texttt{netio} COM interface
	receiving packets from the ethernet device \emph{dev} and
	passing them to the BSD networking code. \emph{send_nio} is
	the \texttt{netio} used by the code to send packets.
	\emph{haddr} contains the MAC address, and \emph{info} the
	device info associated with \emph{dev}.

\end{apidesc}

\begin{apiparm}
	\item[dev] The ethernet device to be used with the interface.
	Note that the \freebsd{} net library will take this reference
	over. 
	\item[out_eif] 
        \emph{*out_eif} points to an \texttt{oskit_freebsd_net_ether_if} 
	structure on success. 
\end{apiparm}

\begin{apiret}
        Returns 0 on success, or an error code specified in
        {\tt <oskit/error.h>}, on error.
\end{apiret}

%%%%%%%%%%%%%%%%%%%%%%%%%%%%%%%%%%%%%%%%%%%%%%%%%%%%%%%%%%%%%%%%%%%%%%%%%%
%
% oskit_freebsd_net_open_first_ether_if
%
\api{oskit_freebsd_net_open_first_ether_if}{find and open 
	first ethernet interface}
\begin{apisyn}
        \cinclude{oskit/net/freebsd.h}

	\funcproto oskit_error_t
		oskit_freebsd_net_open_first_ether_if(
		\outparam struct oskit_freebsd_net_ether_if ** out_eif);
\end{apisyn}
\ostonet

\begin{apidesc}
	This function is a convenience function to find and open
	the first ethernet device\footnote{according to 
	\texttt{osenv_device_lookup} using an \texttt{oskit_etherdev_iid}}
	and to create an associated \texttt{oskit_freebsd_net_ether_if}
	structure.
	\lim{It leaks references to other ethernet devices, if any.}

\end{apidesc}

\begin{apiparm}
	\item[out_eif] 
        \emph{*out_eif} points to an \texttt{oskit_freebsd_net_ether_if} 
	structure on success. 
\end{apiparm}

\begin{apiret}
        Returns 0 on success, or an error code specified in
        {\tt <oskit/error.h>}, on error.
\end{apiret}

%%%%%%%%%%%%%%%%%%%%%%%%%%%%%%%%%%%%%%%%%%%%%%%%%%%%%%%%%%%%%%%%%%%%%%%%%%
%
% oskit_freebsd_net_close_ether_if
%

\api{oskit_freebsd_net_close_ether_if}{close an ethernet interface}
\begin{apisyn}
        \cinclude{oskit/net/freebsd.h}

        \funcproto void 
	oskit_freebsd_net_close_ether_if(struct oskit_freebsd_net_ether_if *eif);
\end{apisyn}
\ostonet

\begin{apidesc}
	The function closes the interface and frees 
	the oskit_freebsd_net_ether_if structure.

	This is currently done by releasing the two \texttt{netio}
	instances and the \texttt{oskit_etherdev_t} instance in 
	\texttt{struct oskit_freebsd_net_ether_if}.
	
\end{apidesc}

\begin{apiparm}
	\item[eif] Interface to be closed.
\end{apiparm}


%%%%%%%%%%%%%%%%%%%%%%%%%%%%%%%%%%%%%%%%%%%%%%%%%%%%%%%%%%%%%%%%%%%%%%%%%%
%
% oskit_freebsd_net_ifconfig
%

\api{oskit_freebsd_net_ifconfig}{configure an interface}
\begin{apisyn}
        \cinclude{oskit/net/freebsd.h}

        \funcproto oskit_error_t  
	oskit_freebsd_net_ifconfig(
		struct oskit_freebsd_net_ether_if *eif,
		char *name, char *ipaddr, char *netmask);
\end{apisyn}
\ostonet

\begin{apidesc}
	This is a temporary convenience function which does the
	setup usually performed by \freebsd{}'s \texttt{ifconfig(8)} command.
	This function is equivalent to the following command:
	\begin{verbatim}
	ifconfig de0 inet 155.99.214.164 link2 netmask 255.255.255.0
	\end{verbatim}
	with \texttt{155.99.214.164} being the IP address to be used
	by the ethernet interface, \texttt{255.255.255.0} the
	netmask of the local subnet, and \texttt{de0} the
	(arbitrary) name of the interface. 

	% Note: God knows why I set the link2/IFF_LINK2 flag. I believe
	% it's a driver option and passed through, so its moot.
	% But I report faithfully what I've done.
	
\end{apidesc}

\begin{apiparm}
	\item[eif] A structure describing the physical interface 
	as returned by \texttt{oskit_freebsd_net_open_ether_if}.
	
	\item[name] The name of the interface. Should be a 3 byte string 
	of the \texttt{"abn"} where \texttt{a} and \texttt{b} are letters
	and \texttt{n} is a number. Use different names for different
	interfaces.

	\item[ipaddr]
	The address to be used by the interface in \texttt{"xxx.xxx.xxx.xxx"}
	notation.
	
	\item[netmask]
	The netmask of the subnet to be used by the interface 
	in \texttt{"xxx.xxx.xxx.xxx"} notation.
\end{apiparm}

\begin{apiret}
        Returns 0 on success, or an error code specified in
        {\tt <oskit/error.h>}, on error.
\end{apiret}


%%%%%%%%%%%%%%%%%%%%%%%%%%%%%%%%%%%%%%%%%%%%%%%%%%%%%%%%%%%%%%%%%%%%%%%%%%
%
% oskit_freebsd_net_add_default_route
%

\api{oskit_freebsd_net_add_default_route}{set a default route}
\begin{apisyn}
        \cinclude{oskit/net/freebsd.h}

        \funcproto oskit_error_t
		oskit_freebsd_net_add_default_route(char *gateway);
\end{apisyn}
\ostonet

\begin{apidesc}
	This function sets a default route.

	\lim{Take a look at the implementation in
	\texttt{freebsd/net/bsdnet_add_default_route.c}. 
	}
\end{apidesc}

\begin{apiparm}
	\item[gateway]
	The IP address of the default gateway to be set in 
	\texttt{"xxx.xxx.xxx.xxx"} notation.
\end{apiparm}

\begin{apiret}
        Returns 0 on success, or an error code specified in
        {\tt <oskit/error.h>}, on error.
\end{apiret}

