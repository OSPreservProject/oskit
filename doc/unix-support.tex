%
% Copyright (c) 1999, 2000 University of Utah and the Flux Group.
% All rights reserved.
% 
% The University of Utah grants you the right to copy and reproduce this
% document or portions thereof for academic, research, evaluation, and
% personal use only, provided that (1) the title page appears prominently,
% and (2) these copyright and permission notices are retained in all copies.
% To arrange for alternate terms, contact the University of Utah at
% csl-dist@cs.utah.edu or +1-801-585-3271.
%
\label{unix}

\section{Introduction}

The ``Unix support'' library provides the substrate to debug and run
certain \oskit{} components in user-mode on Unix-like systems such
as Linux and \freebsd{}.  In particular, you
can use the networking stack with a second network interface, run the
Linux file system code using a regular file instead of a raw
partition, and use general \oskit{} timers.

By default, programs to be run in this environment
are compiled against the \oskit{} headers and linked
against the \oskit{} C library, rather than using the native system
headers and libraries.  This allows more complete \oskit{} kernels to be
run in user mode, but requires linking with an extra ``native OS''
glue library (see section~\ref{sec:nativeos}) in addition to the
Unix support library.

The difference between a standard \oskit{} kernel and one running under
Unix is shown in Figure~\ref{fig-oskitunix}.  When running under Unix,
the low-level kernel and driver support is replaced with glue code
that provides the same interfaces, but does so by 
using Unix system calls instead of accessing the hardware directly.  
The FS and Net components can optionally be replaced; they
can either be Unix support glue, or the original unmodified code.

\psfigure{oskitunix}{%
	Organization of \oskit{} Unix support Framework.
}


\subsection{Supported Platforms}

Support is provided for Linux (ELF with glibc headers), along
with \freebsd{} 2.x (a.out) and 3.x and above (a.out or ELF).
A number of example
programs that use this code can be built in the {\tt examples/unix}
directory.

To run on Linux it's recommended that you use a 2.2.x kernel.  The
2.0.x kernels don't support {\tt sigaltstack} or the 
{\tt SO_SNDLOWAT}, {\tt SO_RECVLOWAT}, {\tt SO_SNDTIMEO} or 
{\tt SO_RCVTIMEO} socket options which are needed for general \oskit{}
support.  If you do run on a 2.0.x kernel, make sure you disable the
{\tt STACKGUARD} code in the pthreads library (currently the only
thing that uses {\tt sigaltstack}) and make sure that your \oskit{}
program can tolerate those socket options being undefined.

Linux doesn't allow the above socket options to be set: they can only
be queried, which may be a problem since the \oskit{} sets them
on occasion.
None of the example programs should rely on this behavior however.


\subsection{Native OS library}
\label{sec:nativeos}

Two problems arise since applications are compiled with the \oskit{}
header files and linked against the \oskit{} libraries.  First, is
the issue of binary-level compatibility between the \oskit{} and any
particular Unix system.  Since the \oskit{} definitions of system data
structures and flags (e.g., the layout of the {\tt stat} structure, or
the value of error codes) are likely to differ from those of the native OS,
there is a need for glue code to map between the two.  Additionally,
the Unix glue layer needs a way to call the underlying Unix system's syscalls.
It cannot invoke them directly by name (e.g., {\tt read}) since those names
are redefined by the \oskit{}.  The \emph{Native OS} library solves both
of these problems, converting syscall arguments, structures and error
return codes as well as providing syscall stubs for the native system calls.
The Native OS library for Linux can be found in
{\tt oskit/examples/unix/linux}; the \freebsd{} version is in
{\tt oskit/examples/unix/freebsd}.


\section{Converting your \oskit{} kernel to run on Unix}

Unless you have a simple \oskit{} kernel (e.g., one that doesn't access the
disk or network), you'll need to modify it to run with
the Unix support code.  The changes aren't pervasive and mostly apply
to the initialization code.  For example, all of the changes to get the
example kernels to run are {\tt \#ifdef}'ed with {\tt OSKIT_UNIX}, so it
should be easy to see what needs to be changed.  To avoid compile-time 
dependencies you can check the global variable {\tt
  oskit_usermode_simulation}.  This variable is used, for example, by
the libstartup libraries to determine which environment they should
initialize for.

There are two different ways to access the network with the Unix
library.  You can either use an emulated network, where socket calls
are mapped directly to system socket calls, or you can talk to an
interface directly.  The emulated network needs to be initialized
before it can be used, so there is a convenience function in
{\tt liboskit_startup.a} to do just that.  You should call
{\tt start_network_native} ({\tt start_network_native_pthreads} if your
kernel uses the pthreads library) where you would otherwise initialize
the network device.  See {\tt socket_bsd}, {\tt http_proxy} and
{\tt disknet} as some examples of how to do this.  

To talk to an interface directly you don't have to make any changes to
your \oskit{} kernel, you need to call {\tt oskit_linux_init_net} to
initialize the network devices.  When you subsequently do an
{\tt osenv_device_lookup} for an ethernet device, you'll get a special one
from the Unix support library that talks directly to an interface.  This
allows you to run your own TCP/IP stack on an interface.  See
section~\ref{subsec:running:network} for more detailed information on
how to run your \oskit{} kernel in this mode.  

As with the network, you can access the disk directly or through an
emulation layer.  To be able to access the standard Unix filesystem
space, you'll need to initialize it by calling {\tt start_fs_native}
which takes a pathname specifying where your \oskit{} filesystem should
be rooted in the Unix filesystem (threaded applications use
{\tt start_fs_native_pthreads}).  The current working directory is
currently set to the same as the root directory.

To access the disk directly {\tt oskit_linux_block_open} is
overridden by the Unix library so that takes a filename instead of a
disk name.  You can specify a regular file or a special file (so as to
access the raw device). 

The {\tt socket_bsd} example kernel can use either method and so is a
good one to look at to see the changes.  The {\tt linux_fs_com} and
{\tt netbsd_fs_com} kernels are good examples of using a regular file under
Unix for your filesystem.


\section{Building for Unix user-mode}

Your link line will also need to change to build with the Unix support
code.  The first thing that will need to change are the start files
({\tt crt0.o} for a.out and {\tt crt*.o} for ELF).  The
Unix support specific start files are installed in the {\tt lib/unix}
subdirectory of your installed \oskit{} tree.

The second step is to put {\tt unix_support.o}
({\tt unix_support_pthreads.o} if you're using pthreads) and the appropriate
Native OS library ({\tt liblinuxsys.a} or {\tt libfreebsdsys.a}) first
in the link line and add {\tt -loskit_unix} to the list of libraries.  

Getting all of this correct is messy and a pain.  Look at the changes
that were made in the example kernel makefiles
({\tt examples/x86/GNUmakerules} vs. {\tt examples/unix/GNUmakerules}).

You can also use the installed GCC driver script
(bin/ix86-oskit-gcc).  This turns your native compiler into a
cross-compiler for the \oskit{} by causing it to use a different ``specs''
file (the specs file can be found in the \oskit{} tree at
{\tt unsupported/scripts/gcc-driver-script-unix.specs}).  Currently the
driver script will only work on an a.out system.


\section{Running \oskit{} kernels with Unix support}

What you get at the end of the build process is a regular Unix executable,
which you can run as normal.  Any arguments that you would normally
pass to your \oskit{} kernel at the boot prompt can be passed on the
command line.


\subsection{Disk}

If you are using the emulated filesystem, then your \oskit{} program will
see the same filesystem space as any other regular Unix executable
It's root and current working directory (``cwd'') are the same as the
parameter you passed to {\tt start_fs_native}.
Thus, unlike a standard Unix program, your cwd is
{\emph NOT} the same as where you ran it.


\subsection{Networking}
\label{subsec:running:network}

If you're using the emulated network then there's nothing you have to
do, things should work the same as a Unix executable.  If you're
accessing an interface directly you'll have to do some more work.
Since \oskit{} kernels assume that they're the only thing that can access
the net, the best way to do networking is to dedicate one or more interface
for use by the \oskit{} on Unix.

There are a couple environment variables that you'll need to set:

\begin{itemize}

\item {\tt ETHERIF} - A list of interfaces to be used by the \oskit{} Unix code.
This list is a comma separated sequence of interface names, where the names are
the same as used by the native {\tt ifconfig} command.  For example, this
might be ``de1,de2'' under \freebsd{} or ``eth1,eth2'' under Linux.
The default value for {\tt ETHERIF} is ``de1'' for \freebsd{} and 
``eth1'' for Linux.

\item {\tt ETHERADDR} - The address of your ethernet card.  The
address should be set in the same format as reported by {\tt ifconfig} for
that interface (i.e., xx:xx:xx:xx:xx:xx, where `x' is a hex digit).  This
variable is only needed if, for some reason, the support code is unable
to determine the address from the interface itself.

\end{itemize}

There are also some differences between running on Linux and \freebsd{}.

\subsubsection{Linux}

First off, you're going to have to run as root to be able to take over
an ethernet interface under Linux.  This makes sense, since if any
user could hijack an interface you'd be in big trouble.

You're going to want to turn arp off on the interface you'll be using.
Also, set the IP address to be a local address (see rfc1918,
{\tt http://www.faqs.org/rfcs/rfc1918.html}).  The following commands
assume that you're using eth1.

\begin{verbatim}
ifconfig eth1 -arp up 10.1.1.1
\end{verbatim}

You can also use {\tt ipfwadm} to make sure the Linux network stack
never sees the packets.  The following commands will add a rule to
deny packets for all protocol types for both input and output.

\begin{verbatim}
ipfwadm -I -a deny -P all -W eth1
ipfwadm -O -a deny -P all -W eth1
\end{verbatim}

\subsubsection{\freebsd{}}

The Unix support library networking code uses the Berkeley Packet Filter
(``BPF'') to get it's packets.  If your kernel isn't compiled with BPF support
you'll have to build a new kernel.  To do this, put {\tt options
bpfilter 2} in your kernel config file and rebuild.  You should set
the number of packet filter instances to at least 2 so that you can
run {\tt tcpdump} along with your \oskit{} kernel.  Normally you'll have to be
root to use the packet filter, but access is only determined by the
permissions on {\tt /dev/bpf}\emph{n}, so you don't necessarily have to.  You'll
have to determine for yourself if the risk of having an exploitable
packet filter is greater than the risk of running your \oskit{} kernels
as root.

As with Linux, you should take measures to ``isolate'' the interfaces you
are working with.  First use a local address and disable arp:

\begin{verbatim}
ifconfig de1 inet 10.1.1.1 netmask 255.0.0.0 -arp
\end{verbatim}

If your kernel is configured with {\tt IPFIREWALL}, you can keep packets
out of the BSD network stack with:

\begin{verbatim}
ipfw add deny all from any to any via de1
\end{verbatim}
