%
% Copyright (c) 1997-2001 University of Utah and the Flux Group.
% All rights reserved.
% 
% The University of Utah grants you the right to copy and reproduce this
% document or portions thereof for academic, research, evaluation, and
% personal use only, provided that (1) the title page appears prominently,
% and (2) these copyright and permission notices are retained in all copies.
% To arrange for alternate terms, contact the University of Utah at
% csl-dist@cs.utah.edu or +1-801-585-3271.
%
\label{clientos}

\section{Introduction}

The ``clientos'' library is responsible for initializing certain portions
of the client operating system in an \oskit{} kernel. Specifically, the
default memory object, the global registry, the C/\posix{} library
environment, and the default console, must all be initialized in order for
the application to work properly. These interfaces are then available to
the various \oskit{} libraries and components, without requiring linktime
dependencies on them.  The C library in particular is dependent on many
external interfaces, which are accessed through its services database. That
services database is given to the C library by the clientos initialization
function once it has finished creating all the necessary objects. For
example, before the application can call the \texttt{malloc} routine in the
C library, the default memory object must be created (see
Section~\ref{oskit-mem}),
the global registry must be created (see Section~\ref{glob-reg}), the memory
object registered in the global registry, and a
reference to the global registry (an instance of a services database) given
to the C library so that it can request a reference to the memory
object. At this point, \texttt{malloc} can now ask the memory object to
allocate the requested amount of memory. Many other interfaces must also be
installed. As the \oskit{} kernel continues to initialize devices and
interfaces, it will hand those objects to the clientos, or in some cases
the objects will be registered in the global registry directly. This
approach enables the separation of the kernel intialization from the
application itself.

\apisec{Initialization}
The clientos is the first library that must be initialized in the
application's \texttt{main} program. Any attempts to allocate memory prior
to this initialization will fail, and the kernel will most likely panic.
To initialize the clientos library:

\api{oskit_clientos_init}{Initialize the client operating system library}
\begin{apisyn}
	\cinclude{oskit/clientos.h}

	\funcproto oskit_error_t oskit_clientos_init(void); \\
	\funcproto oskit_error_t oskit_clientos_init_pthreads(void);
\end{apisyn}
\begin{apidesc}
	Initialize the Client Operating System library. This routine must
	be called immediately in the application's main program. In multi
	threaded applications, use the
	\texttt{oskit_clientos_init_pthreads} interface instead. As an
	example, consider the trivial ``Hello World'' program:
	\begin{codefrag}
	\begin{verbatim}
        void main()
        {
                oskit_clientos_init();

                printf("Hello, World\n");
        }
        \end{verbatim}
	\end{codefrag}

\end{apidesc}
\begin{apiret}
	Returns 0 on success, or an error code specified in
	{\tt <oskit/error.h>}, on error.
\end{apiret}

Several convenience functions are exported from the clientos library that
make it easy to initialize the \texttt{oskit_libcenv} object as the
application continues to initialize devices and interfaces. They are:

\api{oskit_clientos_sethostname}{Set the hostname}
\begin{apisyn}
	\cinclude{oskit/clientos.h}

	\funcproto oskit_error_t
		   oskit_clientos_sethostname(char *hostname, int len);
\end{apisyn}
\begin{apidesc}
	Set the operating system hostname. This is typically called from
	the network initialization code, once the hostname has been
	determined. 
\end{apidesc}
\begin{apiret}
	Returns 0 on success, or an error code specified in
	{\tt <oskit/error.h>}, on error.
\end{apiret}

\api{oskit_clientos_setfsnamespace}{Set the filesystem namespace}
\label{oskit-clientos-setfsnamespace}
\begin{apisyn}
	\cinclude{oskit/clientos.h}

	\funcproto oskit_error_t
		   oskit_clientos_setfsnamespace(oskit_fsnamespace_t *fsn);
\end{apisyn}
\begin{apidesc}
	Set the filesystem namespce object. This is typically called from
	the filesystem initialization code, once the root filesystem has
	been initialized, and the filesystem namespace object created.
\end{apidesc}
\begin{apiret}
	Returns 0 on success, or an error code specified in
	{\tt <oskit/error.h>}, on error.
\end{apiret}

\apisec{C Library Environment}

One of the many COM interfaces that are installed in the services database
received by the C/\posix{} library, is the \texttt{oskit_libcenv} COM
interface.  The \texttt{oskit_libcenv} encapsulates a number of external
interfaces that only the C library needs. For example, before the
application can use any of the filesystem interface calls, it must request
a reference to the filesystem namespace object (see Section~\ref{fsnamespace}),
which handles translation from multi component
pathnames to \texttt{oskit_file} COM objects. The \texttt{oskit_libcenv}
COM interface can be described as an ad-hoc collection of interfaces that
do not belong anyplace else since only the C/\posix{} libraries require
them.

The {\tt oskit_libcenv} COM interface inherits from {\tt IUnknown}, and has
the following additional methods:
\begin{icsymlist}
\item[getfsnamespace]
	Return a reference to the filesystem namespace object.
\item[setfsnamespace]
	Set the filesystem namespace object.
\item[gethostname]
	Return the system hostname.
\item[sethostname]
	Set the hostname.
\item[exit]
	Call the exit function.
\item[setexit]
	Set the exit function.
\item[getconsole]
	Return a reference to the console interface object.
\item[setconsole]
	Set the console interface object.
\item[signals_init]
	Call the signal initialization function.
\item[setsiginit]
	Set the signal initialization function.
\item[sleep_init]
	Initial a sleep record for a future sleep.
\item[sleep]
	Go to sleep until awakened.
\item[wakeup]
	Wakeup a sleeping thread.
\end{icsymlist}

In the descriptions that follow, it should be noted that the accessor
function are intended to be used by the C/POSIX libraries, while the the
functions to modify the object are intended to be used by the clientos
library when setting up the object. 

\api{getfsnamespace}{Get the filesystem namespace}
\begin{apisyn}
	\cinclude{oskit/com/libcenv.h}

	\funcproto oskit_error_t
		   oskit_libcenv_getfsnamespace(oskit_libcenv_t *c,
				oskit_fsnamespace_t **out_fsn);
\end{apisyn}
\begin{apidesc}
	Get the \texttt{oskit_fsnamespace} COM object from the
	\texttt{oskit_libcenv} COM object. The application initialization
	code will typically set the \texttt{oskit_fsnamespace} object when
	it initializes the root filesystem (see
	\texttt{oskit_clientos_setfsnamespace} above). The \posix{} library
	then requests a reference to namespace object when the application
	first tries to use one of the filesystem interface calls in the
	\posix{} library.
\end{apidesc}
\begin{apiparm}
	\item[c]
		The \texttt{oskit_libcenv} object to operate on.
	\item[out_fsn]
		The \texttt{oskit_fsnamespace} object to return.
\end{apiparm}
\begin{apiret}
	Returns 0 on success, or an error code specified in
	{\tt <oskit/error.h>}, on error.
\end{apiret}


\api{setfsnamespace}{Set the filesystem namespace}
\begin{apisyn}
	\cinclude{oskit/com/libcenv.h}

	\funcproto oskit_error_t
		   oskit_libcenv_setfsnamespace(oskit_libcenv_t *c,
				oskit_fsnamespace_t *fsn);
\end{apisyn}
\begin{apidesc}
	Set the \texttt{oskit_fsnamespace} COM object associated with the
	\texttt{oskit_libcenv} COM object. The application initialization
	code will typically set the \texttt{oskit_fsnamespace} object when
	it initializes the root filesystem (see
	\texttt{oskit_clientos_setfsnamespace} above). The \posix{} library
	then requests a reference to namespace object when the application
	first tries to use one of the filesystem interface calls in the
	\posix{} library.
\end{apidesc}
\begin{apiparm}
	\item[c]
		The \texttt{oskit_libcenv} object to operate on.
	\item[fsn]
		The \texttt{oskit_fsnamespace} object to set.
\end{apiparm}
\begin{apiret}
	Returns 0 on success, or an error code specified in
	{\tt <oskit/error.h>}, on error.
\end{apiret}


\api{gethostname}{Get the system hostname}
\begin{apisyn}
	\cinclude{oskit/com/libcenv.h}

	\funcproto oskit_error_t
		   oskit_libcenv_gethostname(oskit_libcenv_t *c,
				char *hostname, int len);
\end{apisyn}
\begin{apidesc}
	Get the system hostname from the \texttt{oskit_libcenv}
	COM object. The application initialization code will typically set
	the hostname when it initializes the network. The \posix{} library
	will then request the hostname as needed by the C library or the
	application program.
\end{apidesc}
\begin{apiparm}
	\item[c]
		The \texttt{oskit_libcenv} object to operate on.
	\item[hostname]
		The returned hostname buffer.
	\item[len]
		The maximum length of the buffer.
\end{apiparm}
\begin{apiret}
	Returns 0 on success, or an error code specified in
	{\tt <oskit/error.h>}, on error.
\end{apiret}


\api{sethostname}{Set the system hostname}
\begin{apisyn}
	\cinclude{oskit/com/libcenv.h}

	\funcproto oskit_error_t
		   oskit_libcenv_sethostname(oskit_libcenv_t *c,
				char *hostname, int len);
\end{apisyn}
\begin{apidesc}
	Set the system hostname associated with the \texttt{oskit_libcenv}
	COM object. The application initialization code will typically set
	the hostname when it initializes the network. The \posix{} library
	will then request the hostname as needed by the C library or the
	application program.
\end{apidesc}
\begin{apiparm}
	\item[c]
		The \texttt{oskit_libcenv} object to operate on.
	\item[hostname]
		The hostname to set.
	\item[len]
		The length of the hostname buffer.
\end{apiparm}
\begin{apiret}
	Returns 0 on success, or an error code specified in
	{\tt <oskit/error.h>}, on error.
\end{apiret}


\api{exit}{Call the exit function}
\begin{apisyn}
	\cinclude{oskit/com/libcenv.h}

	\funcproto void
		   oskit_libcenv_exit(oskit_libcenv_t *c,
					oskit_u32_t status);
\end{apisyn}
\begin{apidesc}
	Call the exit function for the application. This function
	is default to the \oskit{} kernel exit routine, which causes a
	reboot.
\end{apidesc}
\begin{apiparm}
	\item[c]
		The \texttt{oskit_libcenv} object to operate on.
	\item[status]
		The exit value. 
\end{apiparm}
\begin{apiret}
	Returns 0 on success, or an error code specified in
	{\tt <oskit/error.h>}, on error.
\end{apiret}


\api{setexit}{Set the exit function}
\begin{apisyn}
	\cinclude{oskit/com/libcenv.h}

	\funcproto oskit_error_t
		   oskit_libcenv_setexit(oskit_libcenv_t *c,
					void (*exitfunc)(int));
\end{apisyn}
\begin{apidesc}
	Set the exit function for the application.
\end{apidesc}
\begin{apiparm}
	\item[c]
		The \texttt{oskit_libcenv} object to operate on.
	\item[exitfunc]
		The new exit function.
\end{apiparm}
\begin{apiret}
	Returns 0 on success, or an error code specified in
	{\tt <oskit/error.h>}, on error.
\end{apiret}


\api{getconsole}{Get and the console stream object}
\begin{apisyn}
	\cinclude{oskit/com/libcenv.h}

	\funcproto oskit_error_t
		   oskit_libcenv_getconsole(oskit_libcenv_t *c,
				oskit_ttystream_t **out_ttystream);
\end{apisyn}
\begin{apidesc}
	Get the system console object. The system console object
	defaults to a trivial stream implementation that uses the kernel
	console routines.
\end{apidesc}
\begin{apiparm}
	\item[c]
		The \texttt{oskit_libcenv} object to operate on.
	\item[out_ttystream]
		The \texttt{oskit_ttystream} object to return.
\end{apiparm}
\begin{apiret}
	Returns 0 on success, or an error code specified in
	{\tt <oskit/error.h>}, on error.
\end{apiret}


\api{setconsole}{Set the console stream object}
\begin{apisyn}
	\cinclude{oskit/com/libcenv.h}

	\funcproto oskit_error_t
		   oskit_libcenv_setconsole(oskit_libcenv_t *c,
				oskit_ttystream_t *ttystream);
\end{apisyn}
\begin{apidesc}
	Set the system console object.
	Because the console is in use from the moment the
	kernel starts running, changing the console is more complicated
	than just using the setconsole method. The reader is encouraged to
	look at the example kernel in examples/x86/extended/console_tty,
	and the support code in startup/start_console.c.
\end{apidesc}
\begin{apiparm}
	\item[c]
		The \texttt{oskit_libcenv} object to operate on.
	\item[ttystream]
		The \texttt{oskit_ttystream} object to set.
\end{apiparm}
\begin{apiret}
	Returns 0 on success, or an error code specified in
	{\tt <oskit/error.h>}, on error.
\end{apiret}


\api{signals_init}{Call the signal initialization function}
\begin{apisyn}
	\cinclude{oskit/com/libcenv.h}

	\funcproto oskit_error_t
		   oskit_libcenv_signals_init(oskit_libcenv_t *c,
				int (*func)(int, int, void *));
\end{apisyn}
\begin{apidesc}
	Call the signal initialization function for the \posix{}
	library. The signal initialization function defaults to the
	\oskit{} kernel library signal initialization routine (see
	Section~\ref{kern-signals}). The \posix{} library will call this
	routine if
	the application uses any of the \posix{} signal interface
	functions, passing in a function pointer to the callback in the
	\posix{} library that should be invoked when a hardware trap should
	be passed to the application as a signal. The signal initialization
	function can be changed with the \texttt{setsiginit} method,
	although that should be done with caution.
\end{apidesc}
\begin{apiparm}
	\item[c]
		The \texttt{oskit_libcenv} object to operate on.
	\item[func]
		The callback routine for the kernel trap code.
\end{apiparm}
\begin{apiret}
	Returns 0 on success, or an error code specified in
	{\tt <oskit/error.h>}, on error.
\end{apiret}


\api{setsiginit}{Set the signal initialization function}
\begin{apisyn}
	\cinclude{oskit/com/libcenv.h}

	\funcproto oskit_error_t
		   oskit_libcenv_setsiginit(oskit_libcenv_t *c,
				void (*sigfunc)(int (*func)(int,int,void *)));
\end{apisyn}
\begin{apidesc}
	Set the signal initialization function for the \posix{}
	library. The signal initialization function defaults to the
	\oskit{} kernel library signal initialization routine (see
	Section~\ref{kern-signals}). The \posix{} library will call this
	routine if
	the application uses any of the \posix{} signal interface
	functions, passing in a function pointer to the callback in the
	\posix{} library that should be invoked when a hardware trap should
	be passed to the application as a signal.
	Changing the signal initialization function should be done with caution.
\end{apidesc}
\begin{apiparm}
	\item[c]
		The \texttt{oskit_libcenv} object to operate on.
	\item[sigfunc]
		The new signal initialization function.
\end{apiparm}
\begin{apiret}
	Returns 0 on success, or an error code specified in
	{\tt <oskit/error.h>}, on error.
\end{apiret}


\api{sleep_init}{Initialize a sleep record for the sleep/wakeup interface}
\begin{apisyn}
	\cinclude{oskit/com/libcenv.h}

	\funcproto void
		   oskit_libcenv_sleep_init(oskit_libcenv_t *c,
				osenv_sleeprec_t *sleeprec);
\end{apisyn}
\begin{apidesc}
	The sleep/wakeup interface is provided so that the C/\posix{}
	library uses a standard mechanism for giving up control of the CPU,
	in both single and multi threaded applications. The
	\texttt{sleep_init} method initializes a ``sleep record'' structure
	in preparation for the going to sleep waiting for some event to
	occur.  The sleep record is used to avoid races between actually
	going to sleep and the event of interest, and to provide a
	``handle'' on the current activity by which \texttt{wakeup} can
	indicate which process to awaken.
\end{apidesc}
\begin{apiparm}
	\item[c]
		The \texttt{oskit_libcenv} object to operate on.
	\item[sleeprec]
		A pointer to a sleep record, allocated by the caller.
\end{apiparm}


\api{sleep}{Sleep until awakened}
\begin{apisyn}
	\cinclude{oskit/com/libcenv.h}

	\funcproto oskit_error_t
		   oskit_libcenv_sleep(oskit_libcenv_t *c,
				osenv_sleeprec_t *sleeprec,
				struct oskit_timespec *timeout);
\end{apisyn}
\begin{apidesc}
	The \texttt{sleep} method is used to put the caller to sleep.
	The specified sleep record must have been initialized with
	\texttt{sleep_init}.
	An optional \texttt{timeout} value may be supplied. The
	caller will be woken if the timeout expires, and OSKIT_ETIMEDOUT
	will be returned to indicate timeout.
\end{apidesc}
\begin{apiparm}
	\item[c]
		The \texttt{oskit_libcenv} object to operate on.
	\item[sleeprec]
		A pointer to a sleep record, allocated by the caller.
	\item[timeout]
		A timeout value used to bound the length of the sleep
		interval. 
\end{apiparm}
\begin{apiret}
	The \texttt{sleep} function returns 0 if woken up normally,
	otherwise OSKIT_ETIMEDOUT is returned if the timeout expires.
\end{apiret}


\api{wakeup}{Wakeup a sleeping thread}
\begin{apisyn}
	\cinclude{oskit/com/libcenv.h}

	\funcproto void
		   oskit_libcenv_wakeup(oskit_libcenv_t *c,
				osenv_sleeprec_t *sleeprec);
\end{apisyn}
\begin{apidesc}
	The \texttt{wakeup} methods is used to initiate a wakeup
	of any thread which has previously called \texttt{sleep}
	with the indicated sleep record.
\end{apidesc}
\begin{apiparm}
	\item[c]
		The \texttt{oskit_libcenv} object to operate on.
	\item[sleeprec]
		A pointer to a sleep record, allocated by the caller.
\end{apiparm}


\api{clone}{Make a copy of an \texttt{oskit_libcenv} object}
\begin{apisyn}
	\cinclude{oskit/com/libcenv.h}

	\funcproto oskit_error_t
		   oskit_libcenv_clone(oskit_libcenv_t *c,
				oskit_libcenv_t **out_intf);
\end{apisyn}
\begin{apidesc}
	Make a copy of the \texttt{oskit_libcenv} object in \texttt{c}. All
	of the reference counts on the internal objects are adjusted, and a
	new \texttt{oskit_libcenv} object is returned in \texttt{out_intf}. 
	The new object may then modified without affecting the original
	object. 
\end{apidesc}
\begin{apiparm}
	\item[c]
		The \texttt{oskit_libcenv} object to operate on.
	\item[out_intf]
		The new \texttt{oskit_libcenv} object to return.
\end{apiparm}
\begin{apiret}
	Returns 0 on success, or an error code specified in
	{\tt <oskit/error.h>}, on error.
\end{apiret}


\apisec{Memory Interface}
\label{oskit-mem}

The \texttt{oskit_mem} COM interface defines an interface for memory
allocation and deallocation for oskit libraries. As described above, the C
libary \texttt{malloc} routines are implemented in terms of an
\texttt{oskit_mem} object that is created when the clientos is initialized.
This initial memory object is the lowest level memory allocator that is
available to the application. All other memory allocators, such as the
malloc library, the memdebug library (see Section~\ref{memdebug}), and the
device memory allocators (see Section~\ref{dev}), are implemented in terms
of the \texttt{oskit_mem} object that is created when the clientos is
initialized. 

The {\tt oskit_mem} COM interface inherits from {\tt IUnknown}, and has the
following additional methods:
\begin{icsymlist}
\item[alloc]
	Allocate a chunk of memory.
\item[realloc]
	Realloc a chunk of memory.
\item[alloc_aligned]
	Allocate a chunk of memory, subject to an alignment constraint.
\item[free]
	Free a chunk of memory that was allocated with alloc or
	alloc_aligned.
\item[getsize]
	Inquire about the size of a chunk of memory.
\item[alloc_gen]
	Allocate memory with general constraints.
\item[avail]
	Return the amount of free memory.
\end{icsymlist}

\api{alloc}{Allocate a chunk of memory}
\begin{apisyn}
	\cinclude{oskit/com/mem.h}

	\funcproto void *oskit_mem_alloc(oskit_mem_t *m, 
				oskit_u32_t size, oskit_u32_t flags);
\end{apisyn}
\begin{apidesc}
	Allocate a chunk of memory of \texttt{size} bytes, subject to
	various options specified in \texttt{flags}. If successful, a
	pointer to the new chunk of memory is returned. Othersize a NULL
	pointer is returned. The new memory must be deallocated with the
	\texttt{free} method described below. The options that can be
	specifed with the \texttt{flags} parameter are:
	\begin{icsymlist}
	\item[OSKIT_MEM_AUTO_SIZE]
		The memory allocator must keep track of the size of
		allocated blocks allocated using this flag; in this case,
		the value \emph{size} parameter passed in the corresponding
		{\tt free} call is meaningless.  For blocks allocated
		without this flag set, the caller promises to keep track of
		the size of the allocated block, and pass it back to {\tt
		free} on deallocation.
	\item[OSKIT_MEM_NONBLOCKING]
		If set, this flag indicates that the memory allocator must
		not block during the allocation or deallocation operation.
		Any calls to the allocation functions from interrupt
		handlers \emph{must} specify the {\tt OSKIT_MEM_NONBLOCKING}
		flag.
	\item[OSKIT_MEM_PHYS_WIRED]
		Indicates that the must must be non-pageable.  Accesses to
		the returned memory must not fault.
	\item[OSKIT_MEM_PHYS_CONTIG]
		Indicates the underlying physical memory must be
		contiguous.
	\item[OSKIT_MEM_VIRT_EQ_PHYS]
		Indicates the virtual address must \emph{exactly} equal the
		physical address so the driver may use them
		interchangeably.  The {\tt OSKIT_MEM_PHYS_CONTIG} flag must
		also be set whenever this flag is set.
	\item[OSKIT_MEM_ISADMA_MEM]
		This flag applies only to machines with ISA busses or other
		busses that are software compatible with ISA, such as EISA,
		MCA, or PCI\@.  It indicates that the memory allocated must
		be appropriate for DMA access using the system's built-in
		DMA controller.  In particular, it means that the buffer
		must be physically contiguous, must be entirely contained
		in the low 16MB of physical memory, and must not cross a
		64KB boundary.  (By implication, this means that
		allocations using this flag are limited to at most 64KB in
		size.)  The {\tt OSKIT_MEM_PHYS_CONTIG} flag must also be set
		if this flag is set.
	\end{icsymlist}
\end{apidesc}
\begin{apiparm}
	\item[m]
		The memory object to operate on.
	\item[size]
		The size (in bytes) of the chunk to allocate.
	\item[flags]
		Allocation options and constraints.
\end{apiparm}
\begin{apiret}
	Returns a pointer to the new chunk of memory on success, or NULL if
	the request could not be satisfied.
\end{apiret}


\api{realloc}{Reallocate a chunk of memory}
\begin{apisyn}
	\cinclude{oskit/com/mem.h}

	\funcproto void *oskit_mem_realloc(oskit_mem_t *m,
				void *ptr,
				oskit_u32_t oldsize, oskit_u32_t newsize,
				oskit_u32_t flags);
\end{apisyn}
\begin{apidesc}
	Change the size of the previously allocated memory chunk referenced
	by \texttt{ptr}, returning a pointer to the memory chunk. The
	\texttt{flags} must include \texttt{OSKIT_MEM_AUTO_SIZE} if the
	original allocation did, otherwise \texttt{oldsize} must properly
	give the size of the original allocation. If the new size is
	larger, the contents of the newly allocated portion is
	undefined. Any other flags specified in the original allocation
	should necessarily be given as well. If the size of the original
	chunk cannot be changed in place, a new chunk of the proper size
	will be allocated, and the contents of the original chunk copied to
	it.
\end{apidesc}
\begin{apiparm}
	\item[m]
		The memory object to operate on.
	\item[ptr]
		The original memory.
	\item[oldsize]
		The size (in bytes) of the original chunk.
	\item[newsize]
		The size (in bytes) of the new chunk.
	\item[flags]
		Allocation options and constraints.
\end{apiparm}
\begin{apiret}
	Returns a pointer to the chunk of memory on success, or NULL if
	the request could not be satisfied.
\end{apiret}


\api{alloc_aligned}{Allocate a chunk of memory subject to
			alignment constraints}
\begin{apisyn}
	\cinclude{oskit/com/mem.h}

	\funcproto oskit_error_t
		oskit_mem_alloc_aligned(oskit_mem_t *m, 
				oskit_u32_t size,
				oskit_u32_t flags, oskit_u32_t align);
\end{apisyn}
\begin{apidesc}
	Allocate a chunk of memory of \texttt{size} bytes, subject to
	various options specified in \texttt{flags}, and an alignment
	constraint specifed by \texttt{align}. The alignment constraint is
	a power of two integer, which indicates the minimum required
	alignment for the new chunk. If successful, a pointer to the new
	chunk of memory is returned. Othersize a NULL pointer is
	returned. The new memory must be deallocated with the \texttt{free}
	method described below.
\end{apidesc}
\begin{apiparm}
	\item[m]
		The memory object to operate on.
	\item[size]
		The size (in bytes) of the chunk to allocate.
	\item[flags]
		Allocation options and constraints.
	\item[align]
		The alignment constraint for the new chunk.
\end{apiparm}
\begin{apiret}
	Returns a pointer to the chunk of memory on success, or NULL if
	the request could not be satisfied.
\end{apiret}


\api{free}{Free a chunk of memory}
\begin{apisyn}
	\cinclude{oskit/com/mem.h}

	\funcproto void oskit_mem_free(oskit_mem_t *m, void *ptr,
			oskit_u32_t size, oskit_u32_t flags);
\end{apisyn}
\begin{apidesc}
	Deallocate the chunk of memory pointed to by \texttt{ptr}. The
	\texttt{flags} must include \texttt{OSKIT_MEM_AUTO_SIZE} if the
	original allocation did, otherwise \texttt{size} must properly give
	the size of the original allocation.
\end{apidesc}
\begin{apiparm}
	\item[m]
		The memory object to operate on.
	\item[ptr]
		The chunk to deallocate.
	\item[size]
		The size (in bytes) of the original allocation request if
		the chunk was not allocated with \texttt{OSKIT_MEM_AUTO_SIZE}.
	\item[flags]
		Options and constraints.
\end{apiparm}

\api{getsize}{Inquire about the size of a chunk of memory}
\begin{apisyn}
	\cinclude{oskit/com/mem.h}

	\funcproto oskit_u32_t
		oskit_mem_getsize(oskit_mem_t *m, void *ptr);
\end{apisyn}
\begin{apidesc}
	Return the size of the chunk of memory pointed to by \texttt{ptr}.
	The chunk must have been allocated with	\texttt{OSKIT_MEM_AUTO_SIZE}
	for the size to be determined. The returned size may be greater
	than the original size requested, because of rounding done to
	satisfy alignment constraints. 
\end{apidesc}
\begin{apiparm}
	\item[m]
		The memory object to operate on.
	\item[ptr]
		The chunk to inquire about.
\end{apiparm}
\begin{apiret}
	Returns the size of the chunk, or an undefined value if the chunk
	was not allocated with \texttt{OSKIT_MEM_AUTO_SIZE}.
\end{apiret}


\api{alloc_gen}{Allocate a chunk of memory with general constraints}
\begin{apisyn}
	\cinclude{oskit/com/mem.h}

	\funcproto void *oskit_mem_alloc_gen(oskit_mem_t *m,
				oskit_u32_t size,
				oskit_u32_t flags, oskit_u32_t align_bits,
				oskit_u32_t align_ofs);
\end{apisyn}
\begin{apidesc}
	Allocate a chunk of memory meeting various alignment and address
	constraints. It is similar to \texttt{alloc}, but is intended to
	provide an interface more like \texttt{lmm_alloc_gen} (see
	Section~\ref{lmm}).
\end{apidesc}
\begin{apiparm}
	\item[m]
		The memory object to operate on.
	\item[size]
		The size (in bytes) of the chunk to allocate.
	\item[flags]
		Allocation options and constraints.
	\item[align_bits]
		The number of low bits of the returned memory chunk address
		that must match the corresponding bits in \emph{align_ofs}.
	\item[align_ofs]
		The required offset from natural power-of-two alignment.
		If \emph{align_ofs} is zero,
		then the returned memory block will be naturally aligned
		on a $2^{align_bits}$ boundary.
\end{apiparm}
\begin{apiret}
	Returns a pointer to the chunk of memory on success, or NULL if
	the request could not be satisfied.
\end{apiret}


\api{avail}{Return the amount of free memory}
\begin{apisyn}
	\cinclude{oskit/com/mem.h}

	\funcproto oskit_size_t
			oskit_mem_avail(oskit_mem_t *m, oskit_u32_t flags);
\end{apisyn}
\begin{apidesc}
	 Return the amount of free space in the memory object pool.  If
	 \texttt{flags} is non-zero, it should be a memory type flag, which
	 indicates that the return value should be the amount of free space
	 of that type.
\end{apidesc}
\begin{apiparm}
	\item[m]
		The memory object to operate on.
	\item[flags]
		Options and constraints.
\end{apiparm}
\begin{apiret}
	Returns the amount of memory available on success.
\end{apiret}
