%
% Copyright (c) 1996-2001 University of Utah and the Flux Group.
% All rights reserved.
% 
% The University of Utah grants you the right to copy and reproduce this
% document or portions thereof for academic, research, evaluation, and
% personal use only, provided that (1) the title page appears prominently,
% and (2) these copyright and permission notices are retained in all copies.
% To arrange for alternate terms, contact the University of Utah at
% csl-dist@cs.utah.edu or +1-801-585-3271.
%
\label{libmemdebug}
\label{memdebug}

\section{Introduction}

The Memory Debug Utilities Library is a 
set of functions which replace the standard \oskit{}
memory allocation functions, see Section~\ref{memalloc},
of the minimal or FreeBSD C libraries.  The replacement routines detect
problems with memory allocation, and can print out file and
line information, along with a back-trace to the offending 
allocation.

All of the standard functions are covered:
{\tt malloc}, {\tt memalign}, {\tt calloc}, {\tt realloc}, {\tt free},
and {\tt smalloc}, {\tt smemalign}, and {\tt sfree}.

To use the library, just include {\tt -lmemdebug}
on the linker command line
\emph{before} the standard C library (or wherever it is the standard
allocation routines are coming from).

The memdebug library implements a fence-post style {\tt malloc}
debug library.  It
detects the following problems:
\begin{itemize}
\item	{\bf Over-runs and under-runs.} 
	Over-runs and under-runs of allocated memory blocks are
	detected by ``fence-posts'' at each end of every allocated
	block of memory.  
\item	{\bf Allocation/release style mismatches.}
	Mismatches between {\tt malloc} style
	and {\tt smalloc} style allocations and the respective
	{\tt free} function are detected.  This type of error
	is corrected by the library and only a warning is printed. 
\item 	{\bf Memory use after it is {\tt free}'d.}
	Memory is wiped to a recognizable (nonzero) bit pattern
	on allocation and when it is freed,
	to force bugs to show up when memory is used after it is freed.  
	(See below for which values are used where.)
\item	{\bf Incorrect size passed to {\tt sfree}.}
	The {\tt sfree} size is checked against that used
	when the block is created.  
\item	{\bf {\tt free} called on bad blocks.}
	Freeing of blocks that were never allocated or were 
	already released is detected.
\end{itemize}

Whenever a problem is encountered a back-trace (in the form of 
program counter
values) is dumped (back-tracing from the \emph{allocation} of the
memory).  File and line number information from where the allocation
call was made are also printed (if available).  If the failure was
detected in a call to \texttt{free}, the file and line of that call are
printed.  This is called a ``bogosity dump.''

When correctable errors are detected (e.g., 
{\tt sfree}'ing a {\tt malloc}'d block,
or {\tt sfree}'ing with the wrong size block). 
the correct thing will be
done, and the program will continue as normal (except for the
bogosity dump).

Note that file and line number information is only available if
you're using the macro wrappers for the allocators defined in
{\tt memdebug.h}.  The call stack trace is always
available.

One of the shortcomings of the library is that errors are only
detected during explicit calls into the library, and not at the time
that they happen.  The {\tt memdebug_sweep} function will check the
validity of all allocated blocks, and by judiciously sprinkling calls
throughout your code you can narrow down memory trashing problems.
Similarly, the {\tt memdebug_ptrchk} function will run a sanity
check on a single pointer.  Both functions, when printing ``bogosity
dumps'' will also print the file and line at which they were called.

To help detect leaks of unfreed memory, use {\tt memdebug_mark} and 
{\tt memdebug_check}.
{\tt memdebug_mark} tags all allocated blocks, and then 
{\tt memdebug_check} will check for untagged blocks.  In 
this way, you can ``mark'' all blocks as okay 
and at a later point when
all memory allocated after the mark should have been released,
insert a ``check''.  
The library will print a bogosity dump for any allocation
that is untagged.

To help detect accesses after memory is released, or accesses to
uninitialized memory, the library sets all bytes of an allocation to:
\begin{itemize}
\item{\tt 0xaa} after an {\tt smalloc}-style allocation.
\item{\tt 0xbb} after a {\tt malloc}-style allocation.
\item{\tt 0xdd} after {\tt free}.
\item{\tt 0xee} after {\tt sfree}.
\end{itemize}

\subsection{Memdebug Library Configuration}

There are several configuration options in the library-private
{\tt memdebug.h} header file.  The {\tt NO_MEM_FATAL} \texttt{\#define}
controls whether errors in an allocation are fatal (via {\tt panic})
or if they return 0.
The \texttt{\#define} {\tt ALLOW_MORALLY_QUESTIONABLE_PRACTICE}
controls the library's
handling of {\tt malloc(0)} and {\tt free(NULL)}.  While both of these
constructs are technically legal, they usually signal errors in the
caller; the option merely controls whether a message is printed or not.  
The {\tt MALLOC_0_RET_NULL} option controls the behavior of {\tt
malloc(0)}, either returning {\tt NULL} or returning a valid, unique
(per-allocation) pointer.

\subsection{Memdebug Library Internals}
The memdebug library uses two internal routines,
{\tt memdebug_untraced_alloc} and {\tt memdebug_untraced_free}
to actually allocate and free the memory it tracks.
The default implementations of these routines use
% The following is not quite true, they use the libc memory object
% which is inevitably the same as the ``system memory object.''
% But I don't want to try to straighten this out right now...mike
the initial system memory object (see Section~\ref{oskit-mem}).
An implication of this is that the unwrapped {\tt malloc} and the memdebug
wrapped {\tt malloc} \emph{can have different policies}.  This would be the
case if the client OS has provided its own implementation of {\tt malloc}
not based on the system memory object.

When allocating memory on small alignment boundaries, those boundaries
will actually be bumped up to the alignment necessary for the leading
fence-post of the allocation.  Thus, when running under memdebug data
may be aligned at a larger granularity than when running without
memdebug.

All of the routines use {\tt memdebug_printf} to print all output.
This function should always be defined such that it guarantees that
it will never cause any memory to be allocated.
You should override this if you cannot guarantee that {\tt vfprintf} calls 
will not allocate memory.  

\subsection{External Dependencies}

The memdebug library uses several functions, and one global variable
that it does not define.  It
uses {\tt panic} for flagging internal consistency failures, and
{\tt memset} for wiping swaths of memory.
The default implementation of {\tt memdebug_printf} requires 
{\tt vprintf}.

For memory allocation primitives, the memdebug library depends 
on {\tt memdebug_untraced_alloc} and {\tt memdebug_untraced_free}.
As mentioned, the default versions of these depend on
% again, the following is not really correct
the initial system memory object as provided by whatever C library is in use.
Additionally, calls to {\tt mem_lock} and 
{\tt mem_unlock} are used to protect accesses to memdebug's internal
memory lists.
These routines are described in more detail in
the Memory Allocation section of the Minimal C Library chapter,
Section~\ref{memalloc}.) 

\apisec{Debugging versions of standard routines}

The functions listed below are defined as macros in
the header file {\tt oskit/memdebug.h}, they are
also defined as simple wrappers in the library.
The macro versions provide the library with file and
line number information.

They are drop-in replacements for the allocation functions
described in Section~\ref{memalloc}.

\begin{description}
\item[malloc:] \funcproto void *malloc(size_t size);
\item[realloc:] \funcproto void *realloc(void *buf, size_t new_size);
\item[calloc:] \funcproto void *calloc(size_t nelt, size_t elt_size);
\item[memalign:] \funcproto void *memalign(size_t alignment, size_t size);
\item[free:] \funcproto void free(void *buf);
\item[smalloc:] \funcproto void *smalloc(size_t size);
\item[smemalign:] \funcproto void *smemalign(size_t alignment, size_t size);
\item[sfree:] \funcproto void sfree(void *buf, size_t size);
\end{description}

\apisec{Additional Debugging Utilities}

These routines provide 
additional features useful for tracking down
memory leaks and dynamic memory corruption.

\begin{description}
\item[memdebug_mark:] Mark all currently allocated blocks
\item[memdebug_check:] Look for blocks allocated since mark that haven't been freed
\item[memdebug_ptrchk:] Check validity of a pointer's fence-posts
\item[memdebug_sweep:] Check validity of all allocated block's fence-posts
\end{description}

These routines are internal to the memdebug library, but may be
worth overriding in your system.

\begin{description}
\item[memdebug_printf:] A standard printf-style routine that can be
	guaranteed to not allocate any memory.
\item[memdebug_bogosity:] Dumps information about an allocation block
	when an error in the block is detected.
\item[memdebug_store_backtrace:] Stores a back-trace (the call-stack)
	in a provided buffer.
\item[memdebug_untraced_alloc:] Obtain memory from the client OS.
\item[memdebug_untraced_free:] Return memory to the client OS.
\end{description}

% ------------------------
\api{memdebug_mark}{Mark all currently allocated blocks.}
\begin{apisyn}
	\cinclude{oskit/memdebug.h}

	\funcproto void
	memdebug_mark(void);
\end{apisyn}
\begin{apidesc}
	This function walks the list of all allocated objects and
	``marks'' them.  This is useful so that you can determine what
	was allocated before a certain point in your program.

	Objects only have one bit to keep track of marks, so calling
	{\tt memdebug_mark} more than once may not have the effect
	you would like.
\end{apidesc}
\begin{apirel}
	{\tt memdebug_sweep}
\end{apirel}

% ------------------------
\api{memdebug_check}{Look for blocks allocated since mark that haven't been freed.}
\begin{apisyn}
	\cinclude{oskit/memdebug.h}
	
	\funcproto void 
	memdebug_check(void);
\end{apisyn}
\begin{apidesc}
	This functions walks the list of all allocated blocks and for
	each block that is \emph{not} marked (by {\tt memdebug_mark}), 
	it prints a bogosity dump.

	For example, at the beginning of a server loop call
	{\tt memdebug_mark}, then when the server loop is about
	to iterate, call {\tt memdebug_check} to make sure that
	the loop didn't leave any allocated objects lying about.
\end{apidesc}
\begin{apirel}
	{\tt memdebug_bogosity}, {\tt memdebug_mark}
\end{apirel}

% ------------------------
\api{memdebug_ptrchk}{Check validity of a pointer's fence-posts}
\begin{apisyn}
	\cinclude{oskit/memdebug.h}
	
	\funcproto int
	memdebug_ptrcheck(void* ptr);
\end{apisyn}
\begin{apidesc}
	This function runs a host of sanity checks on a given
	pointer.  Of course, these only work if the pointer, {\tt ptr} is
	one returned by a memdebug-wrapped allocator.  For any
	errors a bogosity dump is printed.
\end{apidesc}
\begin{apiparm}
	\item[ptr] A pointer to a memory block allocated by some
		memdebug wrapped allocator.
\end{apiparm}
\begin{apiret}
	Returns -1 if the fence posts are trashed so badly that
	the information in them cannot be trusted.  Returns 1
	if there was a problem detected but it is not ``fatal''.
	Returns 0 if everything is A-okay.
\end{apiret}
\begin{apirel}
	{\tt memdebug_bogosity}
\end{apirel}

% ------------------------
\api{memdebug_sweep}{Check validity of all allocated block's fence-posts}
\begin{apisyn}
	\cinclude{oskit/memdebug.h}

	\funcproto void
	memdebug_sweep(void);
\end{apisyn}
\begin{apidesc}
	This function walks the list of all allocated blocks and calls
	{\tt memdebug_ptrchk} on each entry.
\end{apidesc}
\begin{apirel}
	{\tt memdebug_ptrchk}
\end{apirel}

% ------------------------
\api{memdebug_printf}{A printf-style routine guaranteed not to
allocate memory}
\begin{apisyn}
	\cinclude{oskit/memdebug.h}

	\funcproto int
	memdebug_printf(const~char~*fmt, ...);
\end{apisyn}
\begin{apidesc}
	Works just like standard libc {\tt printf}, but this
	function must be guaranteed to not allocate any memory
	while it runs.
\end{apidesc}
\begin{apiparm}
	\item[fmt] 
		The standard {\tt printf} format string.
	\item[...]
		The standard {\tt printf} arguments for the specific
		format string.
\end{apiparm}
\begin{apiret}
	Returns the standard {\tt printf} return value.
\end{apiret}


% ------------------------
\api{memdebug_bogosity}{Prints a memdebug bogosity message}
\begin{apisyn}
	\cinclude{oskit/memdebug.h}

	\funcproto void
	memdebug_bogosity(memdebug_mhead *head);
\end{apisyn}
\begin{apidesc}
	Prints a bogosity dump given the first fence-post of an
	allocation.  Uses {\tt memdebug_printf} for all output.
	
	This routine is called by all others in the library to
	dump information about an allocation.
\end{apidesc}
\begin{apiparm}
	\item[head]
		The head fence-post for the given allocation.
		Contains the back-trace, file and line number
		information, and allocation-style information.
\end{apiparm}
\begin{apirel}
	{\tt memdebug_printf}
\end{apirel}


% ------------------------
\api{memdebug_store_backtrace}{Stores call-stack trace in provided
buffer}
\begin{apisyn}
	\cinclude{oskit/memdebug.h}

	\funcproto void
	memdebug_store_backtrace(unsigned *backtrace, int max_len);
\end{apisyn}
\begin{apidesc}
	Stores a machine-specific back-trace in the provided buffer.
	In conjunction with the object code and the {\tt nm} utility,
	the back-trace can provide a function call stack.
\end{apidesc}
\begin{apiparm}
	\item[backtrace]
		A buffer of at least {\tt max_len} \texttt{unsigned ints}.
	\item[max_len]
		Size of back-trace buffer.
\end{apiparm}

% ------------------------
\api{memdebug_untraced_alloc}{Obtain memory from the client OS}
\begin{apisyn}
	\cinclude{oskit/memdebug.h}

	\funcproto void *memdebug_untraced_alloc(oskit_u32_t size,
		oskit_u32_t align_bits, oskit_u32_t align_ofs);
\end{apisyn}
\begin{apidesc}
	Obtains memory of the given size and alignment constraints
	from the client OS\@.
	Used by the memdebug library to get the ``raw'' memory that it tracks.
\end{apidesc}
\begin{apiparm}
	\item[size]
		The size (in bytes) of the chunk to allocate.
	\item[align_bits]
		The number of low bits of the returned memory chunk address
		that must match the corresponding bits in \emph{align_ofs}.
	\item[align_ofs]
		The required offset from natural power-of-two alignment.
		If \emph{align_ofs} is zero,
		then the returned memory block will be naturally aligned
		on a $2^{align_bits}$ boundary.
\end{apiparm}

% ------------------------
\api{memdebug_untraced_free}{Return memory from the client OS}
\begin{apisyn}
	\cinclude{oskit/memdebug.h}

	\funcproto void
	memdebug_untraced_free(void *ptr, oskit_u32_t size);
\end{apisyn}
\begin{apidesc}
	Returns the indicated memory with the given size to the client OS\@.
	Used by the memdebug library to free the ``raw'' memory that it tracks.
\end{apidesc}
\begin{apiparm}
	\item[ptr]
		Memory to free.
	\item[size]
		The size (in bytes) of the chunk being freed.
\end{apiparm}

% eof
