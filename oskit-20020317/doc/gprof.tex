%
% Copyright (c) 1998-2000 University of Utah and the Flux Group.
% All rights reserved.
% 
% The University of Utah grants you the right to copy and reproduce this
% document or portions thereof for academic, research, evaluation, and
% personal use only, provided that (1) the title page appears prominently,
% and (2) these copyright and permission notices are retained in all copies.
% To arrange for alternate terms, contact the University of Utah at
% csl-dist@cs.utah.edu or +1-801-585-3271.
%
\label{gprof}

\section{Introduction}

The gprof program and associated libc and kernel routines provide
a mechanism to produce an execution profile of an oskit-based
kernel.  The gprof program is linked right into the kernel,
performing its data reduction and analysis just before the kernel exits
and producing output to the console device.
See gprof(1) for more information.

\section{Caveats}

The application to be profiled must be called ``\texttt{a.out}''.

It is expected that the current behavior of generating ASCII output to
the console device will be changed in the future.  For example, the interface
might be modified to allow specification of an {\tt oskit_stream} object
(Section~\ref{com}) to which the binary ``gmon.out'' data would be written.
Typically, this object would refer to a persistent file or a network
connection with another machine.

See Section~\ref{using-gprof} for line-by-line instructions on using
gprof in the \oskit{}, including some non-obvious linking magic.
You may need to link against additional libraries to use gprof
on your kernel;  If your kernel does not use a bmod filesystem, you'll
need to additionally link against the memfs library.


\section{API reference}
\api{profil}{Enable, disable, or change statistical sampling}
\begin{apisyn}
	\cinclude{oskit/c/sys/gmon.h}

	\cinclude{oskit/c/sys/profile.h}

	\funcproto int profil(char *samples, int size, int offset, int
				scale);
\end{apisyn}
\begin{apidesc}
	This function enables or disables the statistical sampling of
	the program counter for the kernel.  If profiling is enabled,
	at RTC clock tick (see below), the program counter is recorded
	in the {\tt samples} buffer.  This function is most frequently
	called by {\tt moncontrol()}.
\end{apidesc}

\begin{apiparm}
	\item[samples]
		A buffer containing {\tt size} bytes which is divided into 
		a number of bins.  A bin represents a range of
		addresses in which the PC was found when the profiling
		sample was taken.

	\item[size]
		The size in bytes of the samples array.

	\item[offset]
		The lowest address at which PC samples should be
		taken.  In the oskit, this defaults to the location
		of the {\tt _start} symbol.

	\item[scale]
		The scale determines the granularity of the bins.
		A scale of 65536 means each bin gets 2 bytes of
		address range.  A scale of 32768 gives 4 byte, etc.
		A scale value of 0 disables profiling.
\end{apiparm}
\begin{apiret}
	Returns 0 if all is OK, or -1 on error.  Sets {\tt errno} to the
	reason for failure.
\end{apiret}

\api{moncontrol}{enable or disable profiling}
\begin{apisyn}
	\cinclude{oskit/c/sys/gmon.h}

	\cinclude{oskit/c/sys/profile.h}

	\funcproto void moncontrol(int mode);
\end{apisyn}

\begin{apidesc}

If {\tt mode} is non-zero (true), enables profiling.  If {\tt mode} is zero, 
disables profiling.
\end{apidesc}

\begin{apiparm}
\item[mode]  Determines if profiling should be enabled or disabled
\end{apiparm}

\api{monstartup}{Start profiling for the first time}
\begin{apisyn}
	\cinclude{oskit/c/sys/gmon.h}

	\cinclude{oskit/c/sys/profile.h}

	\funcproto void monstartup(unsigned long *lowpc, unsigned long
				*highpc);
\end{apisyn}

\begin{apidesc}
	{\tt monstartup} initiates profiling of the kernel; it should only
	be called once.  Note that by default, {\tt monstartup} is 
	called by {\tt base_multiboot_main} when profiling is enabled
	with {\tt configure}.  If you wish to delay profiling until
	a later time, disable the monstartup call in
	{\tt base_multiboot_main}, and place your own call to
	{\tt monstartup} later in your code.
\end{apidesc}
\begin{apiparm}
	\item[lowpc] The lowest address for which statistics should be
		     collected.  Usually the location of the {\tt _start}
		     symbol.
	\item[highpc] The highest address for which statistics
		     should be collected.  Usually the location of
		     the {\tt etext} (end of text segment) symbol.
\end{apiparm}

\section{Using gprof}
\label{using-gprof}

\begin{enumerate}
\item Configure your sources with --enable-profiling

\item When you link your program, link against:
	\begin{enumerate}

	\item   the \texttt{_p} versions of all libraries you would normally use
        \item   the \texttt{.po} versions of all \texttt{.o} files you would use
             \emph{except} \texttt{crtn.o} and \texttt{multiboot.o} (if you use them)
\sloppy{%
	\item	Insert ``\texttt{-loskit_gprof -loskit_kern_p -loskit_c_p}''
		immediately after the existing ``\texttt{-loskit_kern_p~-loskit_c_p}.''
		That's right, \emph{another} instance of the kern and C libs.
		If you use the FreeBSD C library, do the analogous thing.
		If the above doesn't work, try including the libs more times
		(yes, this is bogus).
}
	\item  Be sure to include the following libraries:
		\begin{itemize}
		\item \texttt{oskit\_dev\_p}
		\item \texttt{oskit\_lmm\_p}
		\item \texttt{oskit_memfs_p}
		\end{itemize}
	\end{enumerate}

\item Run \texttt{mkbsdimage} \emph{multiboot_kernel multiboot_kernel}:\texttt{a.out} \\
	or \\
      \texttt{mkmbimage} \emph{multiboot_kernel multiboot_kernel}:\texttt{a.out}

    This step is necessary so gprof can access the kernel's symbol
    table via the bmodfs.

\item Run the kernel (if created as above, it would be named \texttt{Image}).

\item Profiling output will be spit out at its exit.

\end{enumerate}

\section{Files}
\begin{itemize}
	\item{gprof/*}
		The gprof directory contains the files necessary for
		the gprof program itself.
	\item{libc/gmon/gmon.c}
		contains moncontrol() and monstartup().
	\item{libc/gmon/mcount.c}
		contains the C-language \_mcount routine.
	\item{libc/x86/mcount_md.S}
		contains the asm routines which are linked into
		functions compiled with -pg (mcount for C-language
		functions).  The \_\_mcount routine must be called
		manually by your assembly functions, though this call
		is handled automatically if you use the ENTRY() macro.
	\item{oskit/c/sys/gmon.h}
	\item{oskit/c/sysprofile.h}
		The profiling headers.
	\item{kern/x86/pc/profil.c}
		The profil() system call (architecture-dependent).
\end{itemize}


\section{Changing parameters and other FAQs}


\subsection{The sampling rate}

	The default sampling rate is 8192 Hz, using the RTC as the source
	of the sampling interrupts.  You can adjust this by modifying
	one \#define in gmon.h:

	   Redefine \#define PROFHZ xxxx   to the sampling rate.

	The rate you select must be a power of 2 between 128 and 8192.

\subsection{How can I temporarily disable gprof's output while still linking it in?}

     in base_console.c, change int enable\_gprof = 1;  to = 0.

\subsection{Why isn't there a command line option for it?}

     The FreeBSD boot manager won't pass in the -p flag.

\subsection{Why don't my assembly routines register properly with mcount?}

     You need to hand-code stubs for them which call \_\_mcount.  Sorry.
     The compiler only autogenerates the \_mcount stubs for C
     routines.  The call to \_\_mcount \emph{is} performed for you if
     you use the oskit ENTRY() macro.

\subsection{Why is the call graph wrong when a routine was called from an
     assembly function?}

     If you don't use one of the oskit ENTRY macros, then your function's
     symbol may not be declared properly.  If you want to do it by hand,
     then declare the symbol:
	\begin{verbatim}
            .globl symbol_name
            .type  symbol_name,@function
        \end{verbatim}

     Note that this is taken care of for you by the macros in asm.h if
     you simply declare a function with ENTRY(x) or NON_GPROF_ENTRY(x).

\subsection{What will gprof break?}

Gprof takes over the RTC (irq 8).  If you have code which uses the oskit
interrupt request mechanism to grab irq 8, it won't work.  If your code
just steals irq 8 by replacing the interrupt handler for it, you'll break
gprof.

Gprof installs some atexit handlers for the kernel `main'.  These are
installed in {\tt base_multiboot_main.c}.
