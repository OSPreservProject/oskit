%
% Copyright (c) 2001, 2002 University of Utah and the Flux Group.
% All rights reserved.
% 
% The University of Utah grants you the right to copy and reproduce this
% document or portions thereof for academic, research, evaluation, and
% personal use only, provided that (1) the title page appears prominently,
% and (2) these copyright and permission notices are retained in all copies.
% To arrange for alternate terms, contact the University of Utah at
% csl-dist@cs.utah.edu or +1-801-585-3271.
%
\label{netbsd-uvm}

\section{Introduction}

The NetBSD UVM library provides real \emph{virtual memory} support including
separation of address spaces, demand paging, memory mapped file, memory
protection, and so on.  This library is based on NetBSD's new
virtual memory system UVM (see
\htmladdnormallink{http://www.netbsd.org/Documentation/kernel/uvm.html}
{http://www.netbsd.org/Documentation/kernel/uvm.html}).  Since this library
is derived from a virtual memory system for UNIX, the services provided
are almost the same as for a traditional UNIX virtual memory
system.  With the UVM library, applications are able to manage multiple user
address spaces and map them onto CPU's virtual address space.  Also,
applications can allocate more memory than there is physical memory
available on machines where a disk swap partition is available.

Although the UVM library provides a method to separate kernel and user
address space, it does not offer a way to switch between user and
kernel mode.  Such functions are separately provided by the Simple
Process Library (See Section~\ref{sproc}).

The UVM library depends on the \oskit{}'s POSIX Threads library
(See Section~\ref{pthread}).  The
UVM library is thread safe, multiple threads can manipulate the same user
address space.

\section{Restrictions}

Currently only the x86 implementation is available but it will be not
difficult to port to another architecture where NetBSD has already
ported.

The x86 LDT (Local Descriptor Table) is not supported.

No single threaded implementation is provided.

\section{Virtual Address Space Layout}
\label{netbsd-uvm-address-space}

The address space layout is shown in figure~\ref{uvm-address-layout}.
Since UVM was originally designed for UNIX, the address space layout is similar
to traditional UNIXes; virtual address spaces are separated into two
regions, the kernel and user regions.  The same kernel image appears
in every virtual address space and only one user address space can be
mapped in the user region.  Unlike NetBSD, the kernel does not reside
at higher addresses but remains at its loaded address.  The user region is
above the kernel region.  The base address of a user address space is
fixed at 0x40000000.  A thread running in kernel mode (CPL
(\emph{Current Privilege Level} $<$ 3) can access both the kernel and
the user region and a thread running in user mode (CPL$=$3) can access
only the user region.

At the initialization of the UVM library, all physical memory (0 to
\texttt{phys_max_lmm}) are mapped as $V=R$ (virtual address $=$ real
(physical) address).  The kernel pages are changed to read-only but
all other pages in the $V=R$ range are left read-write since the \oskit{} LMM
library needs write access.  A stack redzone is also created, although
stack overflows are currently fatal since there is not enough support to allow
recovery.

The kernel heap area used for memory allocation (See
Section~\ref{uvm-mem}) starts at 0x30000000.  The area 0x3f800000 to
0x3fffffff is used by the UVM internally for page table manipulation.

\begin{figure}
\centering
\begin{verbatim}
/*
 *  +---------------+ 0x00000000 (OSKIT_UVM_MIN_KERNEL_ADDRESS)     --+    ---+
 *  |               |                                                 |       |
 *  +---------------+                                                 |       |
 *  | OSKit Kernel  |                                                 |       |
 *  +---------------+                                                 | V=R   |
 *  |  BMODs etc.   |                                                 | range |
 *  +---------------+ ?                                               |       |
 *  |               |                                                 |       |
 *  +---------------+            (phys_max_lmm)                     --+       | 
 *  |    unused     |                                                         | kernel
 *  +---------------+ 0x30000000 (OSKIT_UVM_MIN_KERNEL_VIRTUAL_AVAIL)         | region
 *  |  Kernel heap  |                                                         |
 *  +---------------+ (grows upward)                                          |
 *  |               |                                                         |
 *  +---------------+ 0x3f800000                                              |
 *  | Recursive PTE |                                                         |
 *  +---------------+ 0x3fc00000                                              |
 *  | Alternative   |                                                         |
 *  | Recursive PTE |                                                         |
 *  +---------------+ 0x40000000 (OSKIT_UVM_MINUSER_ADDRESS)               ---+
 *  |               |                                                         | user
 *  | User Process  |                                                         | region
 *  |               |                                                         | 
 *  +---------------+ 0xffffffff (OSKIT_UVM_MAXUSER_ADDRESS)               ---+
 */
\end{verbatim}
Symbols starting with OSKIT_UVM_ are defined in \texttt{$<$oskit/uvm.h$>$}.

\caption{UVM Address Space Layout}
\label{uvm-address-layout}
\end{figure}

\section{Memory Interface} 
\label{uvm-mem}
The UVM library also provides an implementation of the \texttt{oskit_mem}
COM interface (See Section~\ref{oskit-mem}).  This implementation
allocates memory from the kernel heap area described above.  The
kernel heap area can grow automatically and might be paged out.  This
interface will be the default memory allocator if the UVM library is
used and thus, the default \texttt{malloc} implementation
uses this interface.

There are some restrictions with this implementation of the \texttt{oskit_mem}
interface:

\begin{itemize}

\item Before the UVM library is initialized, or if flags other than
\texttt{OSKIT_MEM_AUTO_SIZE} and \texttt{OSKIT_MEM_NONBLOCKING} are
specified on a memory allocation request, the implementation allocates
memory from \texttt{malloc_lmm} memory pool.

\item Currently the \texttt{avail} method returns an incorrect value.
\end{itemize}

Also, the UVM library provides the following \texttt{osenv_mem_*} APIs
for the \oskit{} device driver framework (See Section~\ref{dev}).

\begin{itemize}
\item \csymbol{osenv_mem_alloc}
\item \csymbol{osenv_mem_free}
\item \csymbol{osenv_mem_map_phys}
\item \csymbol{osenv_mem_phys_max}
\item \csymbol{osenv_mem_get_phys}
\item \csymbol{osenv_mem_get_virt}
\end{itemize}

\section{Threads}
With the UVM library, all threads are associated with a particular
virtual address space.  Initially all threads are associated with the
kernel-only virtual address space, which maps only the kernel area.
{\tt oskit_uvm_vmspace_set} associates a calling thread with a
specified virtual address space.  Once such an association is made,
that virtual address space is used for all memory access from
the thread.

\section{Page Faults}
\label{uvm-page-faults}
All page faults are processed by the page fault handler ({\tt
oskit_uvm_pfault_handler}).  The page fault handler runs using the
context of the thread that caused the page fault.  If a page fault is
processed successfully, the context of the thread that caused the page
fault is automatically restored.  In case of an error such as
a page protection violation or an invalid memory access,
the application's handler is called if one was installed with
{\tt oskit_uvm_handler_set}.  Otherwise, the SIGSEGV or SIGKILL
signal is raised by {\tt oskit_sendsig}.

The signature of an application fault handler is as follows:
\begin{quote}
\begin{verbatim}
typedef void    (*oskit_uvm_handler)(struct oskit_vmspace *vm,
                                     int signo, struct trap_state *frame);
\end{verbatim}
where 
\begin{description}
        \item[vm]
                The virtual address space in which the page fault occurred.
        \item[signo]
                A UNIX signal number.  Normally, the value will be
                SIGSEGV, but it might also be SIGKILL in the case of
		a memory resource shortage.  This is hint only.
        \item[frame]
    	        A pointer to the stack frame of the page fault handler.
\end{description}

A page fault handler should either map a page of memory at the indicated
virtual address or terminate the thread which caused the page fault.
\end{quote}

\section{Virtual memory system calls}

The UVM library provides implementations of the \texttt{mmap},
\texttt{munmap}, \texttt{mprotect}, and \texttt{madvise} system calls.

\section{API reference}

\api{oskit_uvm_init}{initialize the UVM library}
\begin{apisyn}
        \cinclude{oskit/uvm.h}

        \funcproto void oskit_uvm_init(void);
\end{apisyn}
\begin{apidesc}
        Initialize the UVM library.  It does:

        \begin{itemize}
        \item Reserve 70\% of the available physical memory block
        from the {\tt malloc_lmm} memory pool (See
        Section~\ref{malloc-lmm}) for the use of later memory
        allocation.

        \item Installs a default page fault handler.

        \item Initialize the page table as described in
        Section~\ref{netbsd-uvm-address-space}.

        \item Turn on paging.

        \end{itemize}

        This function must be called after \texttt{pthread_init}
        but before any other threads have been created.
        This is because the \oskit{} UVM library associates
        information with every
        thread and currently the implementation relies on that
        information being set before the thread starts.

\end{apidesc}

\api{oskit_uvm_swap_init}{Initialize the swap subsystem}
\begin{apisyn}
        \cinclude{oskit/uvm.h}

        \funcproto void oskit_uvm_swap_init(void);
\end{apisyn}
\begin{apidesc}
        Initialize the swap subsystem.  This function must be called
        before calling \texttt{oskit_uvm_swap_on}.
\end{apidesc}
\begin{apirel}
        {\tt oskit_uvm_swap_on}, {\tt oskit_uvm_swap_off}
\end{apirel}

\api{oskit_uvm_swap_on}{Add a swap area}
\begin{apisyn}
        \cinclude{oskit/uvm.h}

        \funcproto oskit_error_t oskit_uvm_swap_on(oskit_iunknown_t *iunknown);
\end{apisyn}
\begin{apidesc}
        Add a swap area.  Multiple such areas are possible.
\end{apidesc}
\begin{apiparm}
        \item[iunknown]
                An \emph{oskit_absio_t~*} or \emph{oskit_blkio_t~*} that is
                suitable for use as the swap area.  This object should not be
                accessed directly while the UVM system is active.
\end{apiparm}
\begin{apirel}
        {\tt oskit_uvm_swap_init}, {\tt oskit_uvm_swap_off}, {\tt swapon}
\end{apirel}

\api{oskit_uvm_swap_off}{Remove a swap area}
\begin{apisyn}
        \cinclude{oskit/uvm.h}

        \funcproto oskit_error_t oskit_uvm_swap_off(oskit_iunknown_t *iunknown);
\end{apisyn}
\begin{apidesc}
        Remove a swap area.   Currently this function is broken
        due to a bug in the NetBSD code on which this library is based.
\end{apidesc}
\begin{apirel}
        {\tt oskit_uvm_swap_init}, {\tt oskit_uvm_swap_on}, {\tt swapoff}
\end{apirel}

\api{oskit_uvm_start_pagedaemon}{start the pagedaemon}
\begin{apisyn}
        \cinclude{oskit/uvm.h}

        \funcproto void oskit_uvm_start_pagedaemon(void);
\end{apisyn}
\begin{apidesc}
        Start the page daemon thread.  Must be called after calling
        {\tt oskit_uvm_swap_init}.
\end{apidesc}
\begin{apirel}
        {\tt oskit_uvm_swap_init}
\end{apirel}

\api{oskit_uvm_create}{Create a virtual address space}
\begin{apisyn}
        \cinclude{oskit/uvm.h}

        \funcproto oskit_error_t oskit_uvm_create(oskit_size_t size, \outparam oskit_vmspace_t *out_vm);
\end{apisyn}
\begin{apidesc}
        Create a virtual address space.  The virtual address space
        information is stored in the opaque \texttt{oskit_vmspace_t}
        structure which is defined in \texttt{$<$oskit/uvm.h$>$}. This
        function does not map the created address space as the current
        address space.  Use {\tt oskit_uvm_vmspace_set} to access
        the created virtual address space.

\end{apidesc}
\begin{apiparm}
        \item[size]
                The size of the address space in bytes.
        \item[out_vm]
                A pointer to where the created
                \texttt{oskit_vmspace_t} is stored.  Caller must
                allocate the memory for that \texttt{oskit_vmspace_t}
                structure.
\end{apiparm}
\begin{apirel}
        {\tt oskit_uvm_destroy}
\end{apirel}

\api{oskit_uvm_destroy}{Destroy a virtual address space}
\begin{apisyn}
        \cinclude{oskit/uvm.h}

        \funcproto oskit_error_t oskit_uvm_destroy(oskit_vmspace_t vm);
\end{apisyn}
\begin{apidesc}
        Destroy a virtual address space.  No thread must be associated
        with the specified virtual address space.
\end{apidesc}
\begin{apirel}
        {\tt oskit_uvm_create}
\end{apirel}

\api{oskit_uvm_mmap}{Map an object into memory}
\begin{apisyn}
        \cinclude{oskit/uvm.h}

        \funcproto oskit_error_t oskit_uvm_mmap(oskit_vmspace_t vm,
        \inoutparam oskit_addr_t *addr,
                               oskit_size_t size, int prot, int flags,
                               oskit_iunknown_t *iunknown, oskit_off_t foff);
\end{apisyn}
\begin{apidesc}
        Map an object into a virtual address space.  This is similar to
        \texttt{mmap} but allows the virtual address space affected to be
        explicitly specified.
\end{apidesc}
\begin{apiparm}
        \item[vm]
                The virtual address space to be used for this operation.

        \item[addr]
                On entry, the value pointed to by this parameter is the
                the preferable virtual address to map.  Must be a multiple
                of the page size.  This is a hint only and can be specified
                as zero.  On return, it contains the start address of the
                mapped object.

        \item[size]
                The length to map.  Must be a multiple of the page size.

        \item[prot]
                The desired memory protection.  Bitwise combination of
                PROT_EXEC, PROT_READ, PROT_WRITE or PROT_NONE.

        \item[flags]
                Specifies the type of the mapped object.
                \begin{description}
                  \item[MAP_FIXED]
                    Must use the specified address exactly.
                  \item[MAP_SHARED]
                    Writes change the underlying object; modifications
                    are shared.
                  \item[MAP_PRIVATE]
                    Writes only change our mapping; modifications are private.
                  \item[MAP_ANON]
                    Maps anonymous memory not associated with any specific
                    object.  The {\tt iunknown} parameter may be specified
                    as NULL.
                \end{description}

        \item[iunknown]
                An \emph{oskit_absio_t~*} or \emph{oskit_blkio_t~*} object to
                map.  The object reference is added by the UVM.  In
                the case of multiple threads accessing the same mapped
                object, {\tt iunknown} must be a properly wrapped
                object (see Section~\ref{pthread-wrappers}).

        \item[foff]
                The offset of the object to map.  Must be a multiple
                of the page size.
\end{apiparm}
\begin{apiret}
        Returns 0 on success, or an error code specified in
        \cinclude{oskit/errno.h}, on error.
\end{apiret}
\begin{apirel}
{\tt oskit_uvm_munmap}
\end{apirel}

\api{oskit_uvm_unmap}{Remove a mapping}
\begin{apisyn}
        \cinclude{oskit/uvm.h}

        \funcproto oskit_error_t oskit_uvm_munmap(oskit_vmspace_t vm, oskit_addr_t addr,
                                 oskit_size_t len);
\end{apisyn}
\begin{apidesc}
        Deletes the mapping of the specified address range from the given
	address space.  Causes further references to addresses within the
        range to generate invalid memory references.
\end{apidesc}
\begin{apiparm}
        \item[vm]
                The virtual address space to be used for this operation.

        \item[addr]
                The base address of the region to unmap.  Must be a multiple of
                the page size.

        \item[len]
                The length to unmap.  Must be a multiple of the page size.
\end{apiparm}
\begin{apiret}
        Returns 0 on success, or an error code specified in
        \cinclude{oskit/errno.h}, on error.
\end{apiret}
\begin{apirel}
        {\tt oskit_uvm_mmap}
\end{apirel}

\api{oskit_uvm_mprotect}{Control the protection of pages}
\begin{apisyn}
        \cinclude{oskit/uvm.h}

        \funcproto oskit_error_t oskit_uvm_mprotect(oskit_vmspace_t vm, oskit_addr_t addr,
                                   oskit_size_t len, int prot);
\end{apisyn}
\begin{apidesc}
        Changes the specified pages to have protection \emph{prot}.
\end{apidesc}
\begin{apiparm}
        \item[vm]
                The virtual address space to be used for this operation.

        \item[addr]
                The base address of the region.  Must be a multiple of
                the page size.

        \item[size]
                The length of the region.  Must be a multiple of the page size.

        \item[prot]
                The desired memory protection.
\end{apiparm}
\begin{apiret}
        Returns 0 on success, or an error code specified in
        \cinclude{oskit/errno.h}, on error.
\end{apiret}
\begin{apirel}
        {\tt oskit_uvm_mmap}
\end{apirel}

\api{oskit_uvm_madvise}{Give advise about use of memory}
\begin{apisyn}
        \cinclude{oskit/uvm.h}

        \funcproto oskit_error_t oskit_uvm_madvise(struct~oskit_vmspace *vm, oskit_addr_t addr,
                                  oskit_size_t len, int behav);
\end{apisyn}
\begin{apidesc}
        Gives the UVM a hint of the memory access behavior of the
        application.  Known behaviors are given in
        \cinclude{oskit/c/sys/mman.h}:
\begin{verbatim}
#define MADV_NORMAL     0       /* no further special treatment */
#define MADV_RANDOM     1       /* expect random page references */
#define MADV_SEQUENTIAL 2       /* expect sequential page references */
#define MADV_WILLNEED   3       /* will need these pages */
#define MADV_DONTNEED   4       /* dont need these pages */
#define MADV_SPACEAVAIL 5       /* insure that resources are reserved */
#define MADV_FREE       6       /* pages are empty, free them */
\end{verbatim}
\end{apidesc}
\begin{apiret}
        Returns 0 on success, or an error code specified in
        \cinclude{oskit/errno.h}, on error.
\end{apiret}

\api{oskit_uvm_vmspace_set}{Associate a thread with a virtual address space}
\begin{apisyn}
        \cinclude{oskit/uvm.h}

        \funcproto oskit_vmspace_t      oskit_uvm_vmspace_set(oskit_vmspace_t vm);
        oskit_vmspace_t      \csymbol{oskit_uvm_kvmspace};
\end{apisyn}
\begin{apidesc}
        Associate the calling thread with specified virtual memory
        space.  Once a thread is associated, all memory accesses from
        the thread refer to the associated virtual address space.
        Initially all threads are associated with the virtual address
        space \texttt{oskit_uvm_kvmspace} which maps the kernel region only.
\end{apidesc}
\begin{apiret}
        Returns the previous virtual address space associated with the
        calling thread.
\end{apiret}
\begin{apirel}
        {\tt oskit_uvm_vmspace_get}
\end{apirel}

\api{oskit_uvm_vmspace_get}{Get the virtual address space associated with a thread}
\begin{apisyn}
        \cinclude{oskit/uvm.h}

        \funcproto oskit_vmspace_t      oskit_uvm_vmspace_get(void);
\end{apisyn}
\begin{apidesc}
        Obtain the virtual address space associated with the calling thread.
\end{apidesc}
\begin{apiret}
        Returns the virtual address space associated with the calling thread.
\end{apiret}
\begin{apirel}
        {\tt oskit_uvm_vmspace_set}
\end{apirel}

\api{copyin}{Copy a range of data from user to kernel}
\begin{apisyn}
        \cinclude{oskit/uvm.h}

        \funcproto oskit_error_t
		copyin(const~void *from, void *to, oskit_size_t len);
\end{apisyn}
\begin{apidesc}
	Copy data
	from a user address space into the kernel address space
	if allowed by the current memory protection bits.
        The entire source memory region specified by \texttt{from}
	and \texttt{len} must be accessible for read by the current
	thread's address space.
\end{apidesc}
\begin{apiparm}
	\item[from]
		User-space virtual address to copy from.
	\item[to]
		Kernel-space virtual address to copy into.
	\item[len]
		Length of data to copy.
\end{apiparm}
\begin{apiret}
        Returns 0 on success, \texttt{OSKIT_E_POINTER} if the user memory
	region is not accessible.
\end{apiret}

\api{copyinstr}{Copy a string from user to kernel}
\begin{apisyn}
        \cinclude{oskit/uvm.h}

        \funcproto oskit_error_t
		copyinstr(const~void *from, void *to, oskit_size_t len,
		        \outparam oskit_size_t *lencopied);
\end{apisyn}
\begin{apidesc}
	Copy a zero-terminated string of at most \texttt{len} bytes
	from a user address space into the kernel address space
	if allowed by the current memory protection bits.
        The source string specified by \texttt{from}
	must be accessible for read by the current
	thread's address space.
\end{apidesc}
\begin{apiparm}
	\item[from]
		User-space virtual address to copy from.
	\item[to]
		Kernel-space virtual address to copy into.
	\item[len]
		Maximum bytes to copy.
	\item[lencopied]
		Actual length of string copied, including terminating zero.
\end{apiparm}
\begin{apiret}
        Returns 0 on success, \texttt{OSKIT_E_POINTER} if the user memory
	region is not accessible.
\end{apiret}

\api{copyout}{Copy a range of data from kernel to user}
\begin{apisyn}
        \cinclude{oskit/uvm.h}

        \funcproto oskit_error_t
		copyout(const~void *from, void *to, oskit_size_t len);
\end{apisyn}
\begin{apidesc}
	Copy data
	from the kernel address space into a user address space
	if allowed by the current memory protection bits.
        The entire destination memory region specified by \texttt{to}
	and \texttt{len} must be accessible for write by the current
	thread's address space.
\end{apidesc}
\begin{apiparm}
	\item[from]
		Kernel-space virtual address to copy from.
	\item[to]
		User-space virtual address to copy into.
	\item[len]
		Length of data to copy.
\end{apiparm}
\begin{apiret}
        Returns 0 on success, \texttt{OSKIT_E_POINTER} if the user memory
	region is not accessible.
\end{apiret}

\api{copyinstr}{Copy a string from kernel to user}
\begin{apisyn}
        \cinclude{oskit/uvm.h}

        \funcproto oskit_error_t
		copyoutstr(const~void *from, void *to, oskit_size_t len,
		        \outparam oskit_size_t *lencopied);
\end{apisyn}
\begin{apidesc}
	Copy a zero-terminated string of at most \texttt{len} bytes
	from the kernel address space into a user address space
	if allowed by the current memory protection bits.
        The destination string memory specified by \texttt{to}
	must be accessible for write by the current
	thread's address space.
\end{apidesc}
\begin{apiparm}
	\item[from]
		Kernel-space virtual address to copy from.
	\item[to]
		User-space virtual address to copy into.
	\item[len]
		Maximum bytes to copy.
	\item[lencopied]
		Actual length of string copied, including terminating zero.
\end{apiparm}
\begin{apiret}
        Returns 0 on success, \texttt{OSKIT_E_POINTER} if the user memory
	region is not accessible.
\end{apiret}


\api{oskit_uvm_mem_map_phys}{Map a physical memory range into a virtual address space}
\begin{apisyn}
        \cinclude{oskit/uvm.h}

        \funcproto int          oskit_uvm_mem_map_phys(oskit_addr_t pa, oskit_size_t size,
                                       void **addr, int flags);
\end{apisyn}
\begin{apidesc}
Map the specified physical memory range into a virtual address space.
This call is exactly the same as \texttt{osenv_mem_map_phys}.
\end{apidesc}
\begin{apiparm}
        \item[pa]
                Starting physical address.
        \item[length]
                Amount of memory to map.
        \item[kaddr]
                Kernel virtual address allocated and
                returned by the kernel that maps the
                specified memory.
        \item[flags]
                Memory mapping attributes, as described above.
\end{apiparm}
\begin{apiret}
        Returns 0 on success, non-zero on error.
\end{apiret}
\begin{apirel}
        {\tt osenv_mem_map_phys}
\end{apirel}

\api{oskit_handler_set}{Install a fault handler}
\begin{apisyn}
        \cinclude{oskit/uvm.h}

        \funcproto void         oskit_uvm_handler_set(oskit_vmspace_t vm,
                                      oskit_uvm_handler handler);
\end{apisyn}
\begin{apidesc}
Install a handler that is called when the default page fault handler
fails.  See Section~\ref{uvm-page-faults}.
\end{apidesc}
\begin{apiparm}
        \item[vm]
                The virtual address space where the handler is installed.
        \item[handler]
                The pointer to the handler.
\end{apiparm}
\begin{apirel}
        {\tt oskit_uvm_handler_get}
\end{apirel}
\api{oskit_uvm_handler_get}{Get the fault handler for an address space}
\begin{apisyn}
        \cinclude{oskit/uvm.h}

        \funcproto oskit_uvm_handler    oskit_uvm_handler_get(oskit_vmspace_t vm);
\end{apisyn}
\begin{apidesc}
        Obtain the handler installed for the specified virtual address space.
\end{apidesc}
\begin{apirel}
        {\tt oskit_uvm_handler_set}
\end{apirel}

\api{oskit_uvm_csw_hook_set}{Install a context switch hook}
\begin{apisyn}
        \cinclude{oskit/uvm.h}

        \funcproto void         oskit_uvm_csw_hook_set(void (*hook)(void));
\end{apisyn}
\begin{apidesc}
        Install a hook that is called on every context switch.  This API is
        intended to be used from the Simple Process Library.
\end{apidesc}
\begin{apiparm}
        \item[hook]     
                A pointer to the hook function.
\end{apiparm}
