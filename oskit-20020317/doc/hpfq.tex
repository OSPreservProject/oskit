%
% Copyright (c) 1998 University of Utah and the Flux Group.
% All rights reserved.
% 
% The University of Utah grants you the right to copy and reproduce this
% document or portions thereof for academic, research, evaluation, and
% personal use only, provided that (1) the title page appears prominently,
% and (2) these copyright and permission notices are retained in all copies.
% To arrange for alternate terms, contact the University of Utah at
% csl-dist@cs.utah.edu or +1-801-585-3271.
%
\label{hpfq}

\newcommand{\hpfq}{H-PFQ}


\section{Introduction}

This short section outlines the Hierarchical Packet Fair Queuing
(\hpfq) network link-sharing implementation in the \oskit.
The actual \hpfq{} algorithm implemented is called H-WF$^2$Q and is
described in the SIGCOMM96 paper by Bennet and Zhang.
A working understanding of this paper would be useful in understanding
the use of this library.

Be aware that this is a first-cut implementation and is not thoroughly
tested nor tuned.

This library allows the user to hierarchically schedule outgoing
traffic from different sessions through a single link.
Each session is guaranteed a percentage of its parent's bandwidth,
relative to its siblings.
This is done by creating a scheduling tree where interior nodes
correspond to one-level PFQ schedulers,
and leaf nodes corresond to \texttt{oskit_netio} objects for the
sessions they represent (see Section~\ref{oskit-netio}).
The root node corresponds to the physical link being shared.
The definition of \emph{session} is up to the user,
who controls what is sent to the leaf node \texttt{oskit_netio}'s.
Typical session types are real-time traffic,
traffic from different organizations,
or protocol types.
Note, however, that this more general issue of \emph{packet classification}
is not part of this library.

Currently, only one link can be managed at a time,
because of the \texttt{oskit_pfq_root} global variable.
This will be fixed in a future version.


\section{Configuration}

The \hpfq{} library depends on certain modifications to the \oskit{}
Linux device driver set (Section~\ref{linux-dev}) that are enabled
only when the \oskit{} is configured with the \texttt{--enable-hpfq}
configure option.
This configure option enables \hpfq{}-specific code in the Linux
driver set.

This library will \emph{not} work correctly with an improperly
configured Linux driver set.
Similarly, a Linux driver set configured with \texttt{--enable-hpfq}
will only work correctly for non-\hpfq{} applications if
\texttt{oskit_pfq_root} is \texttt{NULL}.


\section{Usage}
\label{hpfq-usage}

The basic procedure of using this library is to first create a
scheduling hierarchy according to the user's needs,
and then to retrieve the \texttt{oskit_netio}'s from the leaf nodes
for use by the various sessions.
The sessions can then send on these \texttt{oskit_netio}'s and the
data will flow to the root according to the policies and allocations
in place.

There are no restrictions on the format of the data sent
to the leaf \texttt{oskit_netio}'s,
but it must be what the \texttt{oskit_netio} corresponding to the root
expects.
In the common case of \texttt{oskit_netio}'s created by the Linux
driver library,
this data will simply be Ethernet frames.

The creation of the hierarchy is done by first creating a root node
and setting the global variables \texttt{oskit_pfq_root} 
and \texttt{oskit_pfq_reset_path} appropriately
(see Section~\ref{hpfq-api}).
Then various intermediate and/or leaf nodes are created and attached
to the root with appropriate share values.
This process is then repeated as needed for the children of any
intermediate nodes.

In this library, share values are floating point numbers that represent
a percentage of the parent's bandwidth allocated to the child.
For example, a child with share value 0.45 is guaranteed 45\% of the
parent's bandwidth when the child has data to send,
assuming the parent has not over-subscribed its bandwidth.

On a given level of the hierarchy,
only the relative differences between share values is important,
however for simplicity it is recommended that share values on a given
level add up to 1.

A more subtle implication of this relative-differences fact,
is that parents can over-subscribe their bandwidth to their children.
More specifically, there is no guarantee that a session with a share value of,
say 50\%
will actually receive that amount of the parent's bandwidth.
To see this, consider the case of an intermediate node with two children,
each allocated 50\% of the bandwidth.
Another child may be added with a share value of 50\%,
but it will in reality only receive 33\%.
This is more generally termed a problem of \emph{admission control},
and is not currently dealt with in this library.


\section{API reference}
\label{hpfq-api}

The following sections describe the functions exported by the \hpfq{}
library in detail.
All of these functions, as well as the types necessary to use them,
are declared in the header file \texttt{<oskit/hpfq.h>}.

This API reference is split into two parts.
The first part describes the external requirements for the library and the
actual functions exported,
which are basically constructors for \texttt{pfq_sched} and
\texttt{pfq_leaf} objects.
The second part describes the \texttt{pfq_sched} and
\texttt{pfq_leaf} COM interfaces.


\section{External Requirements and Constructors}

This section describes the external requirements of the library
and the actual functions exported,
which consist of ``constructor'' functions to create \texttt{pfq_sched} and
\texttt{pfq_leaf} COM objects.


%%%=====================================================================
\api{oskit_pfq_root}{the root node scheduler}
\begin{apisyn}
	\cinclude{oskit/hpfq.h}

	\texttt{extern pfq_sched_t *\csymbol{oskit_pfq_root};}
\end{apisyn}
\begin{apidesc}
	This variable is not directly used by the \hpfq{} library but
	is used to communicate between the Linux driver set and the
	program using the \hpfq{} library.

	The client of the \hpfq{} library is responsible for setting
	this variable to point to the root node of its scheduling
	hierarchy before any sessions attempt to send to their
	respective leaf \texttt{oskit_netio} objects.

	If this variable is set to the \texttt{NULL} value,
	then the Linux driver library will not call back to the
	\hpfq{} code and thus will behave like a Linux driver set not
	configured for \hpfq{}.

	Note that this implies that only one link can be managed at a time.
	This will be fixed in a future version.
\end{apidesc}

\api{oskit_pfq_reset_path}{pointer to the \texttt{reset_path} function}
\begin{apisyn}
	\cinclude{oskit/hpfq.h}

	\texttt{extern void (*\csymbol{oskit_pfq_reset_path})(pfq_sched_t *);}
\end{apisyn}
\begin{apidesc}
	This variable is not directly used by the \hpfq{} library but
	is used to communicate between the Linux driver set and the
	program using the \hpfq{} library.

	The client of the \hpfq{} library is responsible for setting
	this variable to point to the \texttt{pfq_reset_path} function
	before any sessions attempt to send to their
	respective leaf \texttt{oskit_netio} objects.
\end{apidesc}


%% Basic functions for creating nodes and leaves in the scheduling hierarchy.

%%%=====================================================================
\api{pfq_sff_create_root}{create a root node implementing SFF}
\begin{apisyn}
	\cinclude{oskit/hpfq.h}

	\funcproto oskit_error_t pfq_sff_create_root(oskit_netio_t *link,
				pfq_sched_t **out_sched);
\end{apisyn}
\begin{apidesc}
	Creates a root PFQ node implementing the Smallest Finish time
	First (SFF) scheduling policy.
\end{apidesc}
\begin{apiparm}
	\item[link]
		The link that the scheduling tree intends to manage.
	\item[out_sched]
		A pointer to the \texttt{pfq_sched} object representing
		the root of the hierarchy.

		This can be then used with future \texttt{pfq_sched}
		methods to add children, etc.
\end{apiparm}
\begin{apiret}
	Returns 0 on success, or an error code specified in
	\texttt{<oskit/error.h>}, on error.
\end{apiret}

%%%=====================================================================
\api{pfq_ssf_create_root}{create a root node implementing SSF}
\begin{apisyn}
	\cinclude{oskit/hpfq.h}

	\funcproto oskit_error_t pfq_ssf_create_root(oskit_netio_t *link,
				pfq_sched_t **out_sched);
\end{apisyn}
\begin{apidesc}
	Creates a root PFQ node implementing the Smallest Start time
	First (SSF) scheduling policy.
\end{apidesc}
\begin{apiparm}
	\item[link]
		The link that the scheduling tree intends to manage.
	\item[out_sched]
		A pointer to the \texttt{pfq_sched} object representing
		the root of the hierarchy.

		This can be then used with future \texttt{pfq_sched}
		methods to add children, etc.
\end{apiparm}
\begin{apiret}
	Returns 0 on success, or an error code specified in
	\texttt{<oskit/error.h>}, on error.
\end{apiret}

%%%=====================================================================
\api{pfq_sff_create}{create an intermediate node implementing SFF}
\begin{apisyn}
	\cinclude{oskit/hpfq.h}

	\funcproto oskit_error_t pfq_sff_create(pfq_sched_t **out_sched);
\end{apisyn}
\begin{apidesc}
	Creates an intermediate PFQ node implementing the Smallest Finish
	time First (SFF) scheduling policy.
\end{apidesc}
\begin{apiparm}
	\item[out_sched]
		A pointer to the \texttt{pfq_sched} object representing
		the the created intermediate node.

		This can be then used with future \texttt{pfq_sched}
		methods to add children, etc.
\end{apiparm}
\begin{apiret}
	Returns 0 on success, or an error code specified in
	\texttt{<oskit/error.h>}, on error.
\end{apiret}

%%%=====================================================================
\api{pfq_ssf_create}{create an intermediate node implementing SSF}
\begin{apisyn}
	\cinclude{oskit/hpfq.h}

	\funcproto oskit_error_t pfq_ssf_create(pfq_sched_t **out_sched);
\end{apisyn}
\begin{apidesc}
	Creates an intermediate PFQ node implementing the Smallest Start
	time First (SSF) scheduling policy.
\end{apidesc}
\begin{apiparm}
	\item[out_sched]
		A pointer to the \texttt{pfq_sched} object representing
		the the created intermediate node.

		This can be then used with future \texttt{pfq_sched}
		methods to add children, etc.
\end{apiparm}
\begin{apiret}
	Returns 0 on success, or an error code specified in
	\texttt{<oskit/error.h>}, on error.
\end{apiret}

%%%=====================================================================
\api{pfq_leaf_create}{create a leaf node}
\begin{apisyn}
	\cinclude{oskit/hpfq.h}

	\funcproto oskit_error_t pfq_leaf_create(pfq_leaf_t **out_leaf);
\end{apisyn}
\begin{apidesc}
	Create a leaf PFQ node.

	The \texttt{oskit_netio} that can be used by the session
	corresponding to this leaf can be retreived by calling
	\texttt{pfq_leaf_get_netio}, described elsewhere in this document.
\end{apidesc}
\begin{apiparm}
	\item[out_leaf]
		A pointer to the \texttt{pfq_leaf} object representing
		the the created intermediate node.

		This can be then used with future \texttt{pfq_leaf}
		methods to set the share value, etc.
\end{apiparm}
\begin{apiret}
	Returns 0 on success, or an error code specified in
	\texttt{<oskit/error.h>}, on error.
\end{apiret}


\apiintf{pfq_sched}{Interface to PFQ Schedulers}

This section describes the \texttt{pfq_sched} COM interface to PFQ
scheduler objects.
Note that the \texttt{child} parameter to these methods is declared
as a \texttt{pfq_sched}.
However, \texttt{pfq_leaf} inherits from \texttt{pfq_sched} and thus
may be used as a \texttt{child} parameter when suitably cast.

The \texttt{pfq_sched} interface inherits from \texttt{IUnknown} and
has the following additional methods:
\begin{icsymlist}
	\item[add_child]	Add a child to this node
	\item[remove_child]	Remove a child from this node
	\item[set_share]	Set the bandwidth share given to this node
\end{icsymlist}

%%%=====================================================================
\api{pfq_sched_add_child}{add a child to a root or intermediate node}
\begin{apisyn}
	\cinclude{oskit/hpfq.h}

	\funcproto oskit_error_t pfq_sched_add_child(pfq_sched_t *sched,
				pfq_sched_t *child,
				float share);
\end{apisyn}
\begin{apidesc}
	This method attaches a child \texttt{pfq_sched} object to a
	parent and assigns it an initial share of the parent.
	The share can be later adjusted with the \texttt{set_share}
	method if needed.
\end{apidesc}
\begin{apiparm}
	\item[sched]
		The parent \texttt{pfq_sched} object.
	\item[child]
		The child being added.
	\item[share]
		The initial share value of the child.  See
		Section~\ref{hpfq-usage} for details on share values.
\end{apiparm}
\begin{apiret}
	Returns 0 on success, or an error code specified in
	\texttt{<oskit/error.h>}, on error.
\end{apiret}

%%%=====================================================================
\api{pfq_sched_remove_child}{remove a child from a root or intermediate node}
\begin{apisyn}
	\cinclude{oskit/hpfq.h}

	\funcproto oskit_error_t pfq_sched_remove_child(pfq_sched_t *sched,
				pfq_sched_t *child);
\end{apisyn}
\begin{apidesc}
	This method removes a child from a parent \texttt{pfq_sched} object.
\end{apidesc}
\begin{apiparm}
	\item[sched]
		The parent \texttt{pfq_sched} object losing a child.
	\item[child]
		The child to be orphaned.
\end{apiparm}
\begin{apiret}
	Returns 0 on success, or an error code specified in
	\texttt{<oskit/error.h>}, on error.
\end{apiret}

%%%=====================================================================
\api{pfq_sched_set_share}{allocate a percentage of the parent's bandwidth}
\begin{apisyn}
	\cinclude{oskit/hpfq.h}

	\funcproto oskit_error_t pfq_sched_set_share(pfq_sched_t *sched,
				pfq_sched_t *child,
				float share);
\end{apisyn}
\begin{apidesc}
	This method adjusts the share value of a \texttt{pfq_sched} object.
	See Section~\ref{hpfq-usage} for details on share values.
\end{apidesc}
\begin{apiparm}
	\item[sched]
		The parent \texttt{pfq_sched} object.
	\item[child]
		The child getting their share adjusted.
	\item[share]
		The new share value of the child.
\end{apiparm}
\begin{apiret}
	Returns 0 on success, or an error code specified in
	\texttt{<oskit/error.h>}, on error.
\end{apiret}


\apiintf{pfq_leaf}{Interface to PFQ Leaf Nodes}

This section describes the \texttt{pfq_leaf} COM interface to PFQ
leaf objects.

The \texttt{pfq_leaf} interface inherits from \texttt{pfq_sched} and
the following additional method:
\begin{icsymlist}
	\item[get_netio]
		Get the \texttt{oskit_netio} corresponding to this leaf
\end{icsymlist}

Note that since \texttt{pfq_leaf} inherits from \texttt{pfq_sched},
it may be used in place of a \texttt{pfq_sched} object when suitably cast.

%%%=====================================================================
\api{pfq_leaf_add_child}{add a child to a root or intermediate node}
\begin{apisyn}
	\cinclude{oskit/hpfq.h}

	\funcproto oskit_error_t pfq_leaf_add_child(pfq_leaf_t *sched,
				pfq_sched_t *child,
				float share);
\end{apisyn}
\begin{apidesc}
	This does not make sense for leaf nodes and is thus not implemented.
\end{apidesc}
\begin{apiret}
	Returns 0 on success, or an error code specified in
	\texttt{<oskit/error.h>}, on error.
\end{apiret}

%%%=====================================================================
\api{pfq_leaf_remove_child}{remove a child from a root or intermediate node}
\begin{apisyn}
	\cinclude{oskit/hpfq.h}

	\funcproto oskit_error_t pfq_leaf_remove_child(pfq_leaf_t *sched,
				pfq_sched_t *child);
\end{apisyn}
\begin{apidesc}
	This does not make sense for leaf nodes and is thus not implemented.
\end{apidesc}

%%%=====================================================================
\api{pfq_leaf_set_share}{allocate a percentage of the parent's bandwidth}
\begin{apisyn}
	\cinclude{oskit/hpfq.h}

	\funcproto oskit_error_t pfq_leaf_set_share(pfq_leaf_t *sched,
				pfq_sched_t *child,
				float share);
\end{apisyn}
\begin{apidesc}
	This does not make sense for leaf nodes and is thus not implemented.
\end{apidesc}

%%%=====================================================================
\api{pfq_leaf_get_netio}{get the \texttt{oskit_netio} corresonding to this leaf}
\begin{apisyn}
	\cinclude{oskit/hpfq.h}

	\funcproto oskit_error_t pfq_leaf_get_netio(pfq_leaf_t *leaf,
				oskit_netio_t **out_netio);
\end{apisyn}
\begin{apidesc}
	Retrieves a pointer to an \texttt{oskit_netio} that can be
	used to send data by the session corresponding to this leaf.
\end{apidesc}
\begin{apiparm}
	\item[leaf]
		The leaf who's \texttt{oskit_netio} is of interest.
	\item[out_netio]
		A pointer to the \texttt{oskit_netio} object that can
		be used to send data by the session corresponding to
		this leaf.
\end{apiparm}
\begin{apiret}
	Returns 0 on success, or an error code specified in
	\texttt{<oskit/error.h>}, on error.
\end{apiret}
