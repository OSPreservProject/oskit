%
% Copyright (c) 1997-1998, 2000 University of Utah and the Flux Group.
% All rights reserved.
% 
% The University of Utah grants you the right to copy and reproduce this
% document or portions thereof for academic, research, evaluation, and
% personal use only, provided that (1) the title page appears prominently,
% and (2) these copyright and permission notices are retained in all copies.
% To arrange for alternate terms, contact the University of Utah at
% csl-dist@cs.utah.edu or +1-801-585-3271.
%
The \oskit's interfaces
provide clean, well-defined intra-process or intra-kernel protocols
that can be used to define the interaction
between different modules of an operating system.
For example, the \oskit\ provides a ``block I/O'' interface
for communication between file systems and disk device drivers,
a ``network I/O'' interface
for communication between network device drivers and protocol stacks,
and a file system interface similar to the ``VFS'' interface in BSD\@.
However, the \oskit's interfaces were designed with a number of properties
that make them much more useful as parts of a toolkit
than are comparable traditional OS-level interfaces.
These properties partly stem from the use of
the Component Object Model (COM), described in Chapter~\ref{com}
as the underlying framework in which the interfaces are defined,
and partly just from careful interface design with these properties in mind.
The primary important properties of the \oskit{} interfaces include:

\begin{itemize}
\item	{\bf Extensibility}.
	Anyone can define new interfaces or extend existing ones
	with no need to interact with a centralized authority.
	Components can simultaneously use and export interfaces
	defined by many different sources.
\item	{\bf Simplicity}.
	All of the \oskit's interfaces
	take a minimalist design strategy:
	only the most obvious and fundamental features
	are included in the base interfaces.
	Get it right first;
	frills and optimizations can be added later
	through additional or extended interfaces.
\item	{\bf Full extensibility while retaining interoperability}.
	Adding support in a component for a new or extended interface
	does not cause existing clients to break.
	Similarly, adding support in a client for a new interface
	does not make the client cease to work with existing components.
\item	{\bf Clean separation between components}.
	The clean object model used by the \oskit{} interfaces
	ensures that components do not develop implicit dependencies
	on each other's internal state or implementation.
	Such dependencies can still be introduced explicitly
	when desirable for performance reasons,
	by defining additional specialized interfaces,
	but the object model helps prevent them from developing accidentally.
\item	{\bf Orthogonality of interfaces}.
	Like the \oskit{} components that use them,
	the interfaces themselves are designed to be
	as independent of and orthogonal to each other as possible,
	so that exactly the set of interfaces needed in a particular situation
	can be used without requiring a lot of other loosely related interfaces
	to be implemented or used as well.
\item	{\bf No required standardized infrastructure code}.
	The \oskit{} interfaces can be used
	irrespective of which actual \oskit{} components are used, if any;
	they do not require any fixed ``support libraries'' of any kind
	which all clients must link with in order to use the interfaces.
	This is one important difference
	between the application of COM in the \oskit{}
	versus its original Win32 environment,
	which requires all components to link with a standard ``COM Library.''
\item	{\bf Efficiency}.
	The basic cost of invocation of an \oskit{} COM interface
	is no more than the cost of a virtual function call in C++.
	Through the use of additional specialized interfaces
	even this cost can be eliminated in performance-critical situations.
\item	{\bf Automation}.
	The simplicity and consistent design conventions
	used by the \oskit's interfaces
	make them amenable to use with automated tools,
	such as \htmladdnormallinkfoot{Flick, the Flux IDL Compiler}%
		{http://www.cs.utah.edu/flux/flick/},
	in the future.
\item	{\bf Binary compatibility}.
	The Component Object Model and the \oskit's interfaces
	are designed to support not only source-level
	but binary-level compatibility
	across future generations of components and clients.
\end{itemize}

As with all other parts of the \oskit,
the client is not required to use the \oskit's interfaces
as the primary inter-module interfaces
within the system being designed.
Similarly,
the client may use only some of the interfaces and not others,
or may use the \oskit's interfaces as a base
from which to build more powerful, efficient interfaces
specialized to the needs of the system being developed.
Naturally, since the specific components provided in the \oskit{}
must have \emph{some} interface,
they have been designed to use the standardized \oskit\ interfaces
so that they can easily be used together when desired;
however, the OS developer can choose
whether to adopt these interfaces
as primary inter-module interfaces in the system,
or simply to use them to connect to particular \oskit{} components
that the developer would like to use.

%XXX the COM interfaces are independent of execution environment.

\section{Header File Conventions}

This section describes some specific important properties
of the design and organization of the \oskit{} header files.

\subsection{Basic Structure}

All of the \oskit's public header files
are installed in an \texttt{oskit} subdirectory
of the main installation directory for header files
(e.g., \texttt{/usr/local/include} by default).
Assuming client code is compiled
with the main include directory in its path,
this means that \oskit-specific header files are generally included
with lines of the form `{\tt \#include <oskit/foo.h>}'.
This is also the convention used
by all of the internal \oskit{} components.
Confining all the \oskit{} headers into a subdirectory in this way
allows the client OS to place its own header files
in the same header file namespace with complete freedom,
without worrying about conflicting with \oskit{} header files.

The \oskit{} follows this rule
even for header files with well-known, standardized names:
for example, the ANSI/POSIX header files provided by the minimal C library
(e.g., \texttt{string.h}, \texttt{stdlib.h}, etc.)
are all located in a header file subdirectory called \texttt{oskit/c}.
On the surface this may seem to make it more cumbersome
for the client OS to use these headers and hence the minimal C library,
since for example it would have to `{\tt \#include <oskit/c/string.h>}'
instead of just the standard `{\tt \#include <string.h>}'.
However, this problem can easily be solved
simply by adding the \texttt{oskit/c} subdirectory
to the C compiler's include path
(e.g., add \texttt{-I/usr/local/include/oskit/c} to the GCC command line);
in fact this is exactly what most of the \oskit{} components themselves do.
Furthermore,
strictly confining the \oskit{} headers to the \texttt{oskit} subdirectory,
it makes it possible for the client OS and the \oskit{} itself
to have \emph{several} different sets of ``standard'' header files
coexisting in the same directory structure:
for example, the \oskit{} components derived from Linux or BSD
typically leave \texttt{oskit/c} out of the compiler's include path
and instead use the native OS's header files;
this makes it much easier to incorporate legacy code with minimal changes.

\subsection{Namespace Cleanliness}

A similar namespace cleanliness issue
applies to the actual symbols defined by many the \oskit{} header files.
In particular, all \oskit{} header files defining COM interfaces,
as well as any related header files that they cross-include
such as \texttt{oskit/types.h} and \texttt{oskit/error.h},
\emph{only} define symbols having a prefix of
\texttt{oskit_}, \texttt{OSKIT_}, \texttt{osenv_}, or \texttt{OSENV_}.
This rule allows these headers to be included
along with arbitrary other headers from different environments
without introducing a significant chance of name conflicts.
In fact, the \oskit{} components derived from legacy systems,
such as the Linux driver set and the FreeBSD drivers and TCP/IP stack,
depend on this property,
to allow them to include the \oskit{} headers
defining the COM interfaces they are expected to export,
along with the native Linux or BSD header files
that the legacy code itself relies on.

Once again,
this rule creates a potential problem
for header files whose purpose is to declare standard, well-known symbols,
such as the minimal C library header files.
For example, \texttt{string.h}
clearly should declare \texttt{memcpy} simply as \texttt{memcpy}
and not as \texttt{oskit_memcpy} or somesuch,
since in the latter case the ``C library'' wouldn't be conforming to
the standard C library interface.
However, there are many types, structures, and definitions
that are needed in both the minimal C library headers and the COM interfaces:
for example, both the \texttt{oskit_ttystream} COM interface
and the minimal C library's \texttt{termios.h}
need to have some kind of \texttt{termios} structure;
however, in the former case a disambiguating \texttt{oskit_} prefix is required,
whereas in the latter case such a prefix is not allowed.
Although technically these two \texttt{termios} structures
exist in separate contexts and could be defined independently,
for practical purposes it would be very convenient for them to coincide,
to avoid having to perform unnecessary conversions
in code that uses both sets of headers.
Therefore, the solution used throughout the \oskit{} header files
is to define the non-prefixed versions of equivalent symbols
with respect to the prefixed versions whenever possible:
for example, the \texttt{errno.h} provided by the minimal C library
simply does a `{\tt \#include <oskit/error.h>}'
and then defines all the \texttt{errno} values with lines of the form:

\begin{verbatim}
#define EDOM    OSKIT_EDOM
#define ERANGE  OSKIT_ERANGE
\end{verbatim}

Unfortunately this is not possible for structures
since C does not have a form of \texttt{typedef} statement
for defining one structure tag as an alias for another.
Therefore, the few structures that need to exist in both contexts
(such as the \texttt{termios} structure mentioned above)
are simply defined twice.
Since these structures are generally well-defined
and standardized by ANSI C, \posix{}, or CAE,
they are not expected to change much over time,
so the added maintenance burden should not be significant
and is vastly outweighed by the additional flexibility
provided by the clean separation of namespaces.

\apisec{Common Header Files}

This section describes a few basic header files
that are used throughout the \oskit{}
and are cross-included by many of the other \oskit{} headers.

\api{boolean.h}{boolean type definitions}
\label{oskit-boolean-h}
\begin{apisyn}
	\cinclude{oskit/boolean.h}
\end{apisyn}
\begin{apidesc}
	Defines the fundamental values
	{\tt TRUE} \ttindex{TRUE}
	and
	{\tt FALSE} \ttindex{FALSE}
	for use with the machine-dependent {\tt oskit_bool_t} type.
\end{apidesc}

\api{compiler.h}{compiler-specific macro definitions}
\label{oskit-compiler-h}
\begin{apisyn}
	\cinclude{oskit/compiler.h}
\end{apisyn}
\begin{apidesc}
	Defines a variety of macros used to hide compiler-dependent ways
	of doing things.

	\begin{csymlist}
	\item[OSKIT_BEGIN_DECLS, OSKIT_END_DECLS]
		\ttindex{OSKIT_BEGIN_DECLS} \ttindex{OSKIT_END_DECLS}
		All function prototypes should be surrounded by these macros,
		so that a C++ compiler will identify them as C functions.
	\item[OSKIT_INLINE]
		\ttindex{OSKIT_INLINE}
		Identifies a function as being inline-able.
	\item[OSKIT_PURE]
		\ttindex{OSKIT_PURE}
		Identifies a function as ``pure.''
		A pure function has no side-effects:
		it examines no values other than its arguments
		and changes no values other than its return value.
	\item[OSKIT_NORETURN]
		\ttindex{OSKIT_NORETURN}
		Identifies a function as never returning (e.g., {\tt _exit}).
	\item[OSKIT_STDCALL]
		\ttindex{OSKIT_STDCALL}
		Indicates that a function
		uses an alternative calling convention compatibile with COM\@.
		In particular,
		this option indicates the called function will pop its
		parameters unless there were a variable number of them.
	\end{csymlist}
\end{apidesc}

\api{config.h}{\oskit{} configuration-specific definitions}
\label{oskit-config-h}
\begin{apisyn}
	\cinclude{oskit/config.h}
\end{apisyn}
\begin{apidesc}
	This file is generated by the {\tt configure} program.
	It identifies a number of environment-dependent parameters.
	Currently these are all related to the compiler and assembler
	used to build the \oskit{}.

	\begin{icsymlist}
	\item[HAVE_CR4]
		Defined if the assembler supports the \%cr4 register.
	\item[HAVE_DEBUG_REGS]
		Defined if the assembler supports the debug registers.
	\item[HAVE_P2ALIGN]
		Defined if the assembler supports the {\tt.p2align} pseudo-op.
	\item[HAVE_CODE16]
		Defined if your assembler supports the {\tt.code16} pseudo-op.
	\item[HAVE_WORKING_BSS]
		Defined if your assembler allows {\tt.space} within
		{\tt.bss} segments.
	\item[HAVE_PACKED_STRUCTS]
		Defined if your compiler groks
		{\tt __attribute__((packed))} on structs.
	\item[HAVE_PURE]
		Defined if your compiler groks
		{\tt __attribute__((pure))} on functions.
	\item[HAVE_NORETURN]
		Defined if your compiler groks
		{\tt __attribute__((noreturn))} on functions.
	\item[HAVE_STDCALL]
		Defined if your compiler groks
		{\tt __attribute__((stdcall))} on functions.
	\end{icsymlist}
\end{apidesc}

\api{machine/types.h}{basic machine-dependent types}
\label{oskit-machine-types-h}
\begin{apisyn}
	\cinclude{oskit/machine/boolean.h}
\end{apisyn}
\begin{apidesc}
	This header file defines a number of types
	whose exact definitions are dependent
	on the processor architecture and compiler in use.

	The following set of types
	are guaranteed to be \emph{exactly} the indicated width
	regardless of processor architecture;
	they are used to get around the fact that
	different C compilers assign different meanings
	to the built-in C types such as \texttt{int} and \texttt{long}:
	\begin{icsymlist}
	\item[oskit_s8_t]	Signed 8-bit integer
	\item[oskit_s16_t]	Signed 16-bit integer
	\item[oskit_s32_t]	Signed 32-bit integer
	\item[oskit_s64_t]	Signed 64-bit integer
	\item[oskit_u8_t]	Unsigned 8-bit integer
	\item[oskit_u16_t]	Unsigned 16-bit integer
	\item[oskit_u32_t]	Unsigned 32-bit integer
	\item[oskit_u64_t]	Unsigned 64-bit integer
	\item[oskit_f32_t]	32-bit floating point type
	\item[oskit_f64_t]	64-bit floating point type
	\end{icsymlist}

	The following types depend in various ways
	on the target processor architecture:
	\begin{icsymlist}
	\item[oskit_bool_t]
		This type represents the most efficient integer type
		for storage of simple boolean values;
		on typical architectures it is the smallest integer type
		that the processor can handle with no extra overhead.
	\item[oskit_addr_t]
		This is an unsigned integer type the same size as a pointer,
		which can therefore be used to hold
		virtual or physical addresses and offsets.
	\item[oskit_size_t]
		This is an unsigned integer type
		equivalent to \texttt{oskit_addr_t},
		except that it is generally used to represent
		the size of a data structure or memory buffer,
		or a difference between two \texttt{oskit_addr_t}s.
	\item[oskit_ssize_t]
		This is a signed integer type
		the same size as \texttt{oskit_size}.
	\item[oskit_reg_t]
		This is an unsigned integer type
		of the same size as a general-purpose processor register;
		it is generally but not necessarily always
		equivalent to \texttt{oskit_addr_t}.
	\item[oskit_sreg_t]
		This is a signed integer type
		the same size as \texttt{oskit_reg_t}.
	\end{icsymlist}
\end{apidesc}

\api{types.h}{basic machine-independent types}
\label{oskit-types-h}
\begin{apidesc}
	This header file defines a few basic types
	which are used throughout
	many of the \oskit's COM interfaces.
	These types correspond to standard \posix{} types
	traditionally defined in \texttt{sys/types.h};
	however, this does not mean that the client OS must use these types
	as its standard definitions for the corresponding \posix{} symbols;
	it only means that the client must use these types
	when interacting with \oskit{} components
	through the \oskit's COM interfaces.
	All of the type names are prefixed by \texttt{oskit_},
	for exactly this reason.

	\begin{icsymlist}
	\item[oskit_dev_t]
		Type representing a device number;
		used in the \texttt{oskit_stat} structure
		in the \oskit's file system interfaces.
		Note that the \oskit's file system components themselves
		don't know or care about
		the actual assignment or meaning of device numbers;
		it is up to the client OS to determine
		how these device numbers are used, if at all.
	\item[oskit_ino_t]
		Type representing a file serial number (``inode'' number),
		again used in the \oskit's file system interfaces.
		% XXX should be 64 bit?
	\item[oskit_nlink_t]
		Type representing the link count of a file.
	\item[oskit_pid_t]
		Type representing a process ID\@.
		This type is used in a few COM interfaces,
		such as the \texttt{oskit_ttystream} interface
		which includes \posix-like job control methods.
		Currently the \oskit{} currently does not include
		any process management facilities,
		but this type and the related methods that use it
		are included in case such a facility is added in the future.
	\item[oskit_uid_t]
		Type representing a traditional \posix{} user ID\@.
		The current \oskit{} components
		do not know or care about \posix{} security,
		but for example the NetBSD file system component
		knows how to store and retrieve user IDs
		in BSD file system partitions.
	\item[oskit_gid_t]
		Type representing a traditional \posix{} group ID\@.
		The same considerations apply as for \texttt{oskit_uid_t}.
	\item[oskit_mode_t]
		Type representing a file type and access permissions bit mask;
		again used in the file system interface.
	\item[oskit_off_t]
		Type representing a 64-bit file offset;
		used in the file system interface.
	\item[oskit_wchar_t]
		Unicode ``wide'' character.
	\end{icsymlist}
\end{apidesc}

