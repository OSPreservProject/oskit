%
% Copyright (c) 1996-2002 University of Utah and the Flux Group.
% All rights reserved.
% 
% The University of Utah grants you the right to copy and reproduce this
% document or portions thereof for academic, research, evaluation, and
% personal use only, provided that (1) the title page appears prominently,
% and (2) these copyright and permission notices are retained in all copies.
% To arrange for alternate terms, contact the University of Utah at
% csl-dist@cs.utah.edu or +1-801-585-3271.
%

\ifwww
  \usepackage[tex4ht,html,2]{tex4ht}
\fi
\usepackage{graphicx}
\usepackage{hyperref}

\newcommand{\oskitversion}{0.97}

\ifwww
  \newcommand{\oskittitledraft}{D R A F T \\}
\else
  \newcommand{\oskittitledraft}{\texttt{D~R~A~F~T}\\[5ex]}
\fi
\title{\oskittitledraft
The OSKit: The Flux Operating System Toolkit\\
Version \oskitversion\\
(SnapShot 20020317)}

\author{The Flux Research Group \\[1ex]
	Department of Computer Science \\
	University of Utah \\[1ex]
	{\normalsize Salt Lake City, UT, USA 84112} \\[2ex]
	{\small	\em \htmladdnormallink{http://www.cs.utah.edu/flux/oskit/}%
		{http://www.cs.utah.edu/flux/oskit/}} \\
	{\small \em oskit@flux.cs.utah.edu}
}

% Fix the date so subsequent re-makes don't get a new date with same content.
\date{March 17, 2002}

\makeatletter
%% The `begin' and `end' commands are hooks that we use when producing the WWW
%% version.  See `tex4ht.cfg'.
\gdef\@oskitcopyright@begin{}
\gdef\@oskitcopyright@end{}
\gdef\@oskitcopyright{}
\def\oskitcopyrightbegin#1{\gdef\@oskitcopyright@begin{#1}}
\def\oskitcopyrightend#1{\gdef\@oskitcopyright@end{#1}}
\def\oskitcopyright#1{\gdef\@oskitcopyright{#1}}
\makeatother

\oskitcopyright{%
Copyright \copyright\ 1996-2002, University of Utah.  All rights reserved.
The University of Utah grants you the right to copy and reproduce this
document or portions thereof for academic, research, evaluation, and
personal use only, provided that this title page appears prominently.
To arrange for alternate terms, contact the University at csl-dist@cs.utah.edu
or +1-801-585-3271.}

% We don't need no steenkin' equations - just gimme a working underscore!
\catcode`\_=\active 

\newcommand{\fluke}{Fluke}
\newcommand{\oskit}{OSKit}
\newcommand{\posix}{\textsc{posix}}
\newcommand{\freebsd}{FreeBSD}

\newcommand{\com}[1]{}
\newcommand{\rat}[1]{\footnote{{\bf Rationale:} #1} }
\newcommand{\xxx}[1]{{\em {\bf Fix: } #1}}

\setlength{\textheight}{9in}
\setlength{\textwidth}{6.5in}
\setlength{\topmargin}{-0.5in}
\setlength{\oddsidemargin}{0in}
\setlength{\evensidemargin}{0in}

\newcommand{\psfigure}[2]{%
\begin{figure}%
\centering
\includegraphics[width=0.95\columnwidth]{#1.eps}%
\caption{\footnotesize #2}%
\label{fig-#1}%
\end{figure}%
}

% simple, inline code fragment
% from Leigh
\newenvironment{codefrag}{%
  \begin{trivlist}%
    \leftskip .5in%
    \item[]%
    \begin{minipage}[c]{.9\linewidth}%
  }{%
    \end{minipage}%
    \end{trivlist}}%

% "Icons" flagging processor- and machine-dependent sections
\ifwww
  \newcommand{\pc}{(PC)}
  \newcommand{\intel}{(X86)}
  \newcommand{\intelpc}{(X86 PC)}
  \newcommand{\parisc}{(PA-RISC)}
\else
  \newcommand{\boxed}[1]{{~\vbox{\hrule\hbox{\vrule{\vbox{\vskip 1pt\hbox{%
  \hskip 1pt\rm\tiny\uppercase{#1}\hskip 1pt}\vskip 1pt}}\vrule}\hrule}~}}
  \newcommand{\pc}{\boxed{pc}}
  \newcommand{\intel}{\boxed{x86}}
  \newcommand{\intelpc}{\boxed{x86 pc}}
  \newcommand{\parisc}{\boxed{pa-risc}}
\fi

% Devices.
\def\ostodrv{\begin{apipart}{Direction}OS $\rightarrow$ Component\end{apipart}}
\def\drvtoosn{\begin{apipart}{Direction}Component $\rightarrow$ OS,
					Nonblocking\end{apipart}}
\def\drvtoosb{\begin{apipart}{Direction}Component $\rightarrow$ OS,
					Blocking\end{apipart}}
% Network.
\def\ostonet{\begin{apipart}{Direction}OS $\rightarrow$ 
			Network stack\end{apipart}}
\def\nettoos{\begin{apipart}{Direction}Network stack $\rightarrow$ 
			OS \end{apipart}}
% Filesystem.
\def\ostofs{\begin{apipart}{Direction}client OS $\rightarrow$ filesystem library\end{apipart}}
\def\fstoos{\begin{apipart}{Direction}filesystem library $\rightarrow$ client OS\end{apipart}}


% Macro to generate index entries in a typewriter font
\newcommand{\ttindex}[1]{\index{#1@{\texttt{#1}}}}

% C symbol definitions - leave the space after \bf,
% otherwise latex2html will screw it up.
\newcommand{\csymbol}[1]{{\large\bf #1}\ttindex{#1}}

% C function prototypes
% These sorta work with latex2html, funcproto args are just passed on as is.
\def\funcparamname[#1#2]{\if*#1{\tt *}\funcparamname[#2]\else{\em #1#2}\fi}
\def\funcparams[#1 #2, #3]{%
{\tt #1} \funcparamname[#2]%
\if!#3\else, \funcparams[#3]\fi}
\def\funcparamlistnonempty(#1){(\funcparams[#1, !])}
\def\funcparamlistvoid[#1(void)#2]{%
\if!#2\funcparamlistnonempty#1\else({\tt void})\fi}
\def\funcparamlist(#1){\funcparamlistvoid[(#1)(void)!]}
%
% \funcproto note:
% For functions returning pointers there must be NO space (or newline)
% between the * and the symbol or else the index will wind up with a
% space as the first char of the symbol.
%
\def\funcproto#1 #2#3(#4);{%
\if*#2{\tt #1 #2}\csymbol{#3}\else{\tt #1} \csymbol{#2#3}\fi%
\funcparamlist(#4);}
\def\outparam{{\rm [out] }}
\def\inoutparam{{\rm [in/out] }}

% Lists of function definitions
% This won't work with latex2html.
\newenvironment{funcdefs}{\list{}{\def\makelabel##1{\funcproto##1}}\catcode`\_=\active}{\endlist}

% C structure declarations
\def\cstructelements[ #1 #2; /*#3*/#4]{\if!#3\else~&{\tt#1}&{\tt#2};&/* #3 &*/\\
	\cstructelements[#4]\fi}
\def\cstructhairy#1\{#2\};{\par\noindent{\tt struct }\csymbol{#1} \{\\
	\begin{tabular}{lllll}\cstructelements[#2 ; /*!*/]\end{tabular} \\ \}}
\newcommand{\cstruct}[2]{\cstructhairy #1\{#2\};}

% Miscellaneous C stuff
\newcommand{\cinclude}[1]{\texttt{\#include $<$#1$>$}}
\newenvironment{csymlist}{\begin{list}{}{\def\makelabel##1{{\tt##1}\/:
	}}}{\end{list}}
\newenvironment{icsymlist}{\begin{list}{}{\def
	\makelabel##1{\ttindex{##1}{\tt##1}\/:
	}}}{\end{list}}

% API definitions and sections
\newcommand{\apisec}[1]{\newpage\section{#1}}
\newcommand{\apiintf}[2]{\newpage\section{{\tt #1}: #2}\ttindex{#1}}
% This works pretty well: gives a little extra space over just \subsection.
\newcommand{\api}[2]{\pagebreak[1]\subsection{{\tt #1}: #2}\ttindex{#1}}
\newenvironment{apipart}[1]{\subsubsection*{{\sc#1}} \begin{quote}}{\end{quote}}
\newenvironment{apisyn}{\begin{apipart}{Synopsis}}{\end{apipart}}
\newenvironment{apidesc}{\begin{apipart}{Description}}{\end{apipart}}
\newenvironment{apiparm}{\begin{apipart}{Parameters}%
	\begin{list}{}{\def\makelabel##1{{\em##1}\/: }}}{%
	\end{list}\end{apipart}}
\newenvironment{apiret}{\begin{apipart}{Returns}}{\end{apipart}}
\newenvironment{apierr}{\begin{apipart}{Errors}}{\end{apipart}}
\newenvironment{apirel}{\begin{apipart}{Related Information}}{\end{apipart}}
\newenvironment{apidep}{\begin{apipart}{Dependencies}\begin{csymlist}}{%
	\end{csymlist}\end{apipart}}

% Define a copyright environment that won't waste a lot of paper...
\newenvironment{copyrightEnv}%
	{\tt \obeylines \noindent \footnotesize \catcode`\&=11}%
	{\dotfill\vskip 6pt}

% Redefine the "theindex" env so it makes a TOC entry.  This is verbatim from
% book.cls with one line added.
\ifwww\else
\makeatletter
\renewenvironment{theindex}
               {\if@twocolumn
                  \@restonecolfalse
                \else
                  \@restonecoltrue
                \fi
                \columnseprule \z@
                \columnsep 35\p@
                \twocolumn[\@makeschapterhead{\indexname}]%
                \addcontentsline{toc}{chapter}{\indexname}%
                \@mkboth{\MakeUppercase{\indexname}}%
                        {\MakeUppercase{\indexname}}%
                \thispagestyle{plain}\parindent\z@
                \parskip\z@ \@plus .3\p@\relax
                \let\item\@idxitem}
               {\if@restonecol\onecolumn\else\clearpage\fi}
\makeatother
\fi

%% Redefine `\maketitle' to include our copyright notice.  This is almost
%% identical to the `titlepage' version of `\maketitle' from the LaTeX2e
%% standard `book.cls' file.
%%
\makeatletter
\renewcommand{\maketitle}{\begin{titlepage}%
  \let\footnotesize\small
  \let\footnoterule\relax
  \let \footnote \thanks
  \null\vfil
  \vskip 60\p@
  \begin{center}%
    {\LARGE \@title \par}%
    \vskip 3em%
    {\large
     \lineskip .75em%
      \begin{tabular}[t]{c}%
        \@author
      \end{tabular}\par}%
      \vskip 1.5em%
    {\large \@date \par}%       % Set date in \large size.
  \end{center}\par
  \vfill                        % <<< new
  \noindent%                    % <<< new
  {\small%                      % <<< new
   \@oskitcopyright@begin%      % <<< new
   \@oskitcopyright%            % <<< new
   \@oskitcopyright@end}%       % <<< new
  \@thanks
  \vfil\null
  \end{titlepage}%
  \setcounter{footnote}{0}%
  \global\let\thanks\relax
  \global\let\maketitle\relax
  \global\let\@thanks\@empty
  \global\let\@author\@empty
  \global\let\@date\@empty
  \global\let\@title\@empty
  \global\let\title\relax
  \global\let\author\relax
  \global\let\date\relax
  \global\let\and\relax
}
\makeatother

%%%%%%%%%%%%%%%%%%%%%%%%%%%%%%%%%%%%%%%%%%%%%%%%%%%%%%%%%%%%%%%%%%%%%%%%%%%%%%%

%% End of file.

