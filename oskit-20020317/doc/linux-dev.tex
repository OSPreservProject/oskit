%
% Copyright (c) 1996-1998, 2000, 2001 University of Utah and the Flux Group.
% All rights reserved.
% 
% The University of Utah grants you the right to copy and reproduce this
% document or portions thereof for academic, research, evaluation, and
% personal use only, provided that (1) the title page appears prominently,
% and (2) these copyright and permission notices are retained in all copies.
% To arrange for alternate terms, contact the University of Utah at
% csl-dist@cs.utah.edu or +1-801-585-3271.
%
\label{linux-dev}

The Linux device driver library consists of various native Linux
device drivers coupled with glue code to export the \oskit{}
interfaces such as blkio, netio, and bufio (See Chapter~\ref{oskit-io}).
See the source file \texttt{linux/dev/README} for a list of devices and
their status.

The header files \texttt{oskit/dev/linux_ethernet.h} and
\texttt{oskit/dev/linux_scsi.h} determine which network and SCSI
drivers are compiled into \texttt{liboskit_linux_dev.a}.
Those files also influence driver probing;
see the \texttt{oskit_linux_init} routines below.


\apisec{Initialization and Registration}

There are several ways to initalize this library.
One can either initialize all the compiled-in drivers
(\texttt{oskit_\-linux_\-init_\-devs},
initialize a specific class of drivers
(\texttt{oskit_\-linux_\-init_\-ethernet}),
or initialize specific drivers
(e.g., \texttt{oskit_\-linux_\-init_\-scsi_\-ncr53c8xx}).

These initialization functions initialize various glue code and
register the appropriate device(s) in the device tree,
to be probed with \texttt{oskit_dev_probe}.

\api{oskit_linux_init_devs}{Initialize and register all known drivers}
\begin{apisyn}
	\cinclude{oskit/dev/linux.h}

	\funcproto void oskit_linux_init_devs(void);
\end{apisyn}
\begin{apidesc}
	This function initializes and registers all known drivers.
	The known drivers are: the IDE disk driver, and all drivers
	listed in the \texttt{<oskit/dev/linux_ethernet.h>} and
	\texttt{<oskit/dev/linux_scsi.h>} files.

	Once drivers are registered, their devices may be probed
	via \texttt{oskit_dev_probe}.
\end{apidesc}


\api{oskit_linux_init_net}{Initialize and register known network drivers}
\begin{apisyn}
	\cinclude{oskit/dev/linux.h}

	\funcproto void oskit_linux_init_net(void);
\end{apisyn}
\begin{apidesc}
	This function initializes and registers all available network drivers.
	Currently this means Ethernet drivers only,
	but in the future there may be other network drivers supported such
	as Myrinet.
	The known Ethernet drivers are listed in the
	\texttt{<oskit/dev/linux_ethernet.h>} file.

	Once drivers are registered, their devices may be probed
	via \texttt{oskit_dev_probe}.
\end{apidesc}


\api{oskit_linux_init_ethernet}{Initialize and register known Ethernet network drivers}
\begin{apisyn}
	\cinclude{oskit/dev/linux.h}

	\funcproto void oskit_linux_init_ethernet(void);
\end{apisyn}
\begin{apidesc}
	This function initializes and registers all available Ethernet
	network drivers.
	The known Ethernet drivers are listed in the
	\texttt{<oskit/dev/linux_ethernet.h>} file.

	Once drivers are registered, their devices may be probed
	via \texttt{oskit_dev_probe}.
\end{apidesc}


\api{oskit_linux_init_scsi}{Initialize and register known SCSI disk drivers}
\begin{apisyn}
	\cinclude{oskit/dev/linux.h}

	\funcproto void oskit_linux_init_scsi(void);
\end{apisyn}
\begin{apidesc}
	This function initializes and registers all available SCSI disk drivers.
	The known SCSI drivers are listed in the
	\texttt{<oskit/dev/linux_scsi.h>} file.

	Once drivers are registered, their devices may be probed
	via \texttt{oskit_dev_probe}.
\end{apidesc}


\api{oskit_linux_init_ide}{Initialize and register known IDE disk drivers}
\begin{apisyn}
	\cinclude{oskit/dev/linux.h}

	\funcproto void oskit_linux_init_ide(void);
\end{apisyn}
\begin{apidesc}
	This function initializes and registers all available IDE disk drivers.
	There is currently only one IDE driver.

	Once drivers are registered, their devices may be probed
	via \texttt{oskit_dev_probe}.
\end{apidesc}


\api{oskit_linux_init_scsi_\emph{name}}{Initialize and register a specific SCSI disk driver}
\begin{apisyn}
	\cinclude{oskit/dev/linux.h}

	\funcproto void oskit_linux_init_scsi_\emph{name}(void);
\end{apisyn}
\begin{apidesc}
	This function initializes and registers a specific SCSI disk driver.
	The \emph{name} must be one from the \texttt{name} field of the drivers
	listed in the
	\texttt{<oskit/dev/linux_scsi.h>} file.

	Once drivers are registered, their devices may be probed
	via \texttt{oskit_dev_probe}.
\end{apidesc}


\api{oskit_linux_init_ethernet_\emph{name}}{Initialize and register a specific Ethernet network driver}
\begin{apisyn}
	\cinclude{oskit/dev/linux.h}

	\funcproto void oskit_linux_init_ethernet_\emph{name}(void);
\end{apisyn}
\begin{apidesc}
	This function initializes and registers a specific Ethernet
	network driver.
	The \emph{name} must be one from the \texttt{name} field of the drivers
	listed in the
	\texttt{<oskit/dev/linux_ethernet.h>} file.

	Once drivers are registered, their devices may be probed
	via \texttt{oskit_dev_probe}.
\end{apidesc}


\apisec{Obtaining object references}

Once the desired drivers are initialized, registered, and probed,
one can obtain references to their \texttt{blkio}, \texttt{netio}, etc
interfaces (See Chapter~\ref{oskit-io}) two different ways.

The first way is to look them up via their Linux name, e.g., ``sd0'' for
a SCSI disk, or ``eth0'' for a Ethernet device.
This is described here as it is specific to Linux.

The second, and preferred, way is to use \texttt{osenv_\-device_\-lookup}
to find a detected device with the desired interface, such as
\texttt{oskit_\-etherdev_\-iid} (See Chapter~\ref{libfdev}).


% We only describe oskit_linux_block_open because that is the only
% one that is actually used in the examples.

\api{oskit_linux_block_open}{Open a disk given its Linux name}
\begin{apisyn}
	\cinclude{oskit/dev/linux.h}

	\funcproto oskit_error_t oskit_linux_block_open(const char *name,
					unsigned flags,
					\outparam oskit_blkio_t **out_io);
\end{apisyn}
\begin{apidesc}
	This function takes a Linux name of a disk,
	e.g., ``sd0'' or ``wd1'',
	and returns an \texttt{oskit_blkio_t} that can be used to access
	the device.

	The \texttt{oskit_blkio} interface is described in
	Chapter~\ref{oskit-io}.
\end{apidesc}
\begin{apiparm}
	\item[name]
		The Linux name of the device
		e.g., ``sd0'' or ``wd1''.

	\item[flags]
		Formed by or'ing the following values:
		\begin{description}
		\item[OSKIT_DEV_OPEN_READ]
		\item[OSKIT_DEV_OPEN_WRITE]
		\item[OSKIT_DEV_OPEN_ALL]
		\end{description}

	\item[out_io]
		Upon success, is set to point to an \texttt{oskit_blkio_t}
		that can be used to access the device.
\end{apiparm}
\begin{apiret}
	Returns 0 on success, or an error code specified in
	{\tt <oskit/error.h>}, on error.
\end{apiret}


\api{oskit_linux_block_open_kdev}{Open a disk given its Linux kdev}


\api{oskit_linux_netdev_find}{Open a netcard given its Linux name}


\api{oskit_linux_net_open}{Open a netcard given its Linux name}


\newpage
\emph{The rest of this chapter is very incomplete.
Some of the internal details of the Linux driver emulation are described,
but not the aspects relevant for typical use of the library.}

\section{Introduction}

XXX

\emph{Much of the data here on Linux device driver internals
is out-of-date with respect to the newer device drivers that are now
part of the \oskit{}.  This section documents drivers from Linux
1.3.6.8 or earlier; the current \oskit{} drivers are from Linux
2.2.12, so parts of this section are likely no longer correct.}


XXX Library can be used either as one component
or can be used to produce many separate components,
depending on how it is used.

\section{Partially-compliant Drivers}

There are a number of assumptions made by \emph{some} drivers:
if a given assumption is not met by the OS using the framework,
then the drivers that make the assumption will not work,
but other drivers may still be usable.
The specific assumptions made by each partially-compliant driver
are listed in a table in the appropriate section below;
here is a summary of the assumptions some of the drivers make:
\begin{itemize}
\item	Kernel memory can be allocated from interrupt handlers.
\item	Drivers can allocate contiguous chunks of physical memory
	larger than one page.
\item	(x86) Drivers can allocate memory specifically in the low 16MB
	of memory accessible to the PC's built-in DMA controller.
\item	Drivers can sleep uninterruptibly.
\item	Drivers can access the clock timer and DMA registers directly.
\item	``Poll-and-Yield:'' polls for short
	periods of time and yields the CPU without explicitly going to sleep.

\end{itemize}

\section{Internals}

The following sections document all the variables and functions that Linux
drivers can refer to.
These variables and functions are provided by the glue code
supplied as part of the library,
so this information should not be needed for normal use of the library
under the device driver framework.
However, they are documented here
for the benefit of those working on this library
or upgrading it to new versions of the Linux drivers,
or for those who wish to ``short-cut'' through the framework
directly to the Linux device drivers in some situations,
e.g., for performance reasons.


\subsection{Namespace Management Rules}

\sloppy{%
For an outline of our namespace management conventions,
see Section 4.7.2 in our SOSP paper,
\htmladdnormallink{http://www.cs.utah.edu/flux/papers/index.html\#SOSKIT}%
		  {http://www.cs.utah.edu/flux/papers/index.html\#SOSKIT}.
}

% doc/papers/oskit-sosp16.ps.   ....sorry, this has copyright problems.

\com{%
XXX make this fit.

To address preprocessor and type conflicts and the like,
the rule is that imported OS code _only_ includes its native headers
plus any ``glue'' headers that we write specifically for that OS\@.
In other words, the Linux drivers should definitely never see <oskit/page.h>
or any of the other \oskit\ header files outside the linux_dev package.
(The specific problem above is that Linux's <asm-i386/page.h>
was changed to include <oskit/page.h>, which it shouldn't do.)
The fdev glue code specific to a particular OS (e.g., linux/dev/*.c)
can include that OS's headers, the glue headers specific to that OS,
_and_ the generic fdev headers defining the fdev interfaces (fdev/*.h),
but must not include any other \oskit\ headers.
Clients of these device drivers and all other code outside a driver set
must never include any of the native OS headers used within the driver set;
this is already enforced by the native OS include directory
being a private subdirectory in the driver set (e.g., linux_fdev/include/)
rather than a subdirectory of the public header file directory (oskit/).
By following these rules, imported OS symbols never touch
any symbols defined by the \oskit\ or the client OS,
with the exception of the fdev/* headers defining the interface between the two;
and _all_ of the symbols in these fdev headers are prefixed with `fdev_'.

To address linker namespace conflicts,
the rule is that _all_ global linker symbols used in a given driver set
must be given a common prefix for that driver set, e.g., `fdev_linux_'.
Global symbols defined by the glue code for that driver set,
either to be publicly exported to clients or merely for internal use,
must be named according to this convention from the outset.
(For example, see oskit_linux_dev_init() or oskit_linux_block_open().)
Global symbols defined or used by the native OS code being imported
must be given this common prefix using appropriate preprocessor magic.
Specifically, I've created (but not yet checked in)
a Linux glue-code header file called `global.h'
that simply consists of a bunch of #define's to add this prefix.
In linux/dev/GNUmakerules, I've added a magic GCC flag
causing this header file to be included at the top of _all_ linux/dev sources,
including both the glue code sources and the raw Linux sources.
This means that both the Linux glue code and the Linux code itself
can use the plain Linux symbols and be oblivious to the fact
that they're getting prefixed.
The initial global.h I created is consistent with the set
of global variables and functions Shantanu documented in linux-dev.tex.
Naturally, there are quite a few more that need to be added,
some because they were originally missed or appeared in the Linux-2.0 upgrade,
others because they are both defined and used by the drivers themselves
and thus never appeared to us as ``undefined symbols'' on linker runs
(e.g., the SCSI globals used by the various SCSI-related drivers).
Although it may sound like a pain to keep this list up-to-date,
it is really quite trivial to find any symbols we missed using nm and grep.
Also, a lot of the remaining global symbols look like they don't need to be,
and we can probably address them by sending patches to the Linux people.

>you said that the files belonging to BSD should only see the
>BSD header files, and that only the glue code would see both.
>
>However, take the case of `struct proc'. Clearly, to implement
>sleep/wakeup, this struct must be enhanced with a osenv_sleeprec_t,
>thus requiring to double definitions or to pull in the dev header
>tree in all files including sys/proc.h (practically all.)

Good point.  Yes, it's probably OK to pull in fdev header files
from modified BSD include files (we do that in Linux already),
since all of the fdev symbols defined should have a disambiguating prefix.
I guess the rule will have to be that the native BSD code
only _directly_ includes native BSD header files.

}%com

\subsection{Variables}

\begin{icsymlist}
\item[current]

This is a global variable that points to the state for the current
process.  It is mostly used by drivers to set or clear the
interruptible state of the process.

\item[jiffies]

Many Linux device drivers depend on a global variable called {\tt jiffies},
which in Linux contains a clock tick counter
that is incremented by one at each 10-millisecond (100Hz) clock tick.
The device drivers typically read this counter while polling a device
during a (hopefully short) interrupt-enabled busy-wait loop.
Although a few drivers
take the global clock frequency symbol {\tt HZ} into account
when determining timeout values and such,
most of the drivers just used hard-coded values
when using the {\tt jiffies} counter for timeouts,
and therefore assume that {\tt jiffies} increments
``about'' 100 times per second.

\item[irq2dev_map]

This variable is an array of pointers to network device structures.
The array is indexed by the interrupt request line (IRQ) number.
Linux network drivers use it in interrupt handlers to find the
interrupting network device given the IRQ number passed to them by the
kernel.

\item[blk_dev]

This variable is an array of ``struct blk_dev_struct'' structures.  It
is indexed by the major device number.  Each element contains the I/O
request queue and a pointer to the I/O request function in the
driver.  The kernel queues I/O requests on the request queue, and calls
the request function to process the queue.

\item[blk_size]

This variable is an array of pointers to integers.  It is indexed by
the major device number.  The subarray is indexed by the minor device
number.  Each cell of the subarray contains the size of the device in
1024 byte units.  The subarray pointer can be NULL, in which case, the
kernel does not check the size and range of an I/O request for the device.

\item[blksize_size]

This variable is an array of pointers to integers.  It is indexed by
the major device number.  The subarray is indexed by the minor device
number.  Each cell of the subarray contains the block size of the
device in bytes.  The subarray can be NULL, in which case, the kernel
uses the global definition BLOCK_SIZE (currently 1024 bytes) in its
calculations.

\item[hardsect_size]

This variable is an array of pointers to integers.  It is indexed by
the major device number.  The subarray is indexed by the minor device
number.  Each cell of the subarray contains the hardware sector size of the
device in bytes.  If the subarray is NULL, the kernel uses 512 bytes
in its calculations.

\item[read_ahead]

This variable is an array of integers indexed by the major device
number.  It specifies how many sectors of read-ahead the kernel should
perform on the device.  The drivers only initialize the values in this
array; the Linux kernel block buffer code is the actual user of these
values.

\item[wait_for_request]

The Linux kernel uses a static array of I/O request structures.  When
all I/O request structures are in use, a process sleeps on this
variable.  When a driver finishes an I/O request and frees the I/O
request structure, it performs a wake up on this variable.

\item[EISA_bus]

If this variable is non-zero, it indicates that the machine has an
EISA bus.  It is initialized bye the Linux kernel prior to device
configuration.

\item[high_memory]

This variable contains the address of the last byte of physical memory
plus one.  It is initialized by the Linux kernel prior to device
configuration.

\item[intr_count]

This variable gets incremented on entry to an interrupt handler,
and decremented on exit.  Its purpose is let driver code determine if it
was called from an interrupt handler.

\item[kstat]

This variable contains Linux kernel statistics counters.  Linux
drivers increment various fields in it when certain events occur.

\item[tq_timer]

Linux has a notion of ``bottom half'' handlers.  These handlers have a
higher priority than any user level process but lower priority than
hardware interrupts.  They are analogous to software interrupts in
BSD\@.  Linux checks if any ``bottom half'' handlers need to be run when
it is returning to user mode.  Linux provides a number of lists of
such handlers that are scheduled on the occurrence of specific events.
{\tt tq_timer} is one such list.  On every clock interrupt,
Linux checks if any handlers are on this list, and if there are,
immediately schedules the handlers to run.

\com{%
	This is only used by dev.c which is not a driver.  I am not sure
	if I should include it here.

\item[bh_base]

This variable is an array of ``struct bh_struct'' elements.  It is
indexed by global constants defined in <linux/interrupt.h>.  Each
element of this array contains a pointer to a function and a pointer
to an opaque data item which is passed to the function when it is
invoked.  The array holds the list of software interrupt handlers.
Driver can directly add an entry in the array, and the Linux kernel
arranges for the handlers to be invoked at the appropriate junction.
}

\item[timer_active]

This integer variable indicates which of the timers in {\tt timer_table}
(described below) are active.  A bit is set if the timer is active,
otherwise it is clear.

\item[timer_table]

This variable is an array of ``struct timer_struct'' elements.  The
array is index by global constants defined in <linux/timer.h>.  Each
element contains the duration of timeout, and a pointer to a function
that will be invoked when the timer expires.

\item[system_utsname]

This variable holds the Linux version number.  Some drivers check the
kernel version to account for feature differences between different
kernel releases.
\end{icsymlist}

\subsection{Functions}

\begin{icsymlist}
\item[autoirq_setup]
\funcproto int autoirq_setup(int waittime);

This function is called by a driver to set up for probing IRQs.
The function attaches a handler on each available IRQ, waits
for \emph{waittime} ticks, and returns a bit mask of IRQs available
IRQs.  The driver should then force the device to generate an interrupt.

\item[autoirq_report]
\funcproto int autoirq_report(int waittime);

This function is called by a driver after it has programmed the device
to generate an interrupt.  The function waits \emph{waittime} ticks, and
returns the IRQ number on which the device interrupted.  If no
interrupt occurred, 0 is returned.

\item[register_blkdev]
\funcproto int register_blkdev(unsigned major, const char *name,
				struct~file_operations *fops);

This function registers a driver for the major number \emph{major}.
When an access is made to a device with the specified major number,
the kernel accesses the driver through the operations vector \emph{fops}.
The function returns 0 on success, non-zero otherwise.

\item[unregister_blkdev]
\funcproto int unregister_blkdev(unsigned major, const char *name);

This function removes the association between a driver and the major
number \emph{major}, previously established by {\tt register_blkdev}.
The function returns 0 on success, non-zero otherwise.

\item[getblk]
\funcproto struct~buffer_head *getblk(kdev_t dev, int block, int size);

This function is called by a driver to allocate a buffer \emph{size} bytes in
length and associate it with device \emph{dev}, and block number \emph{block}.

\item[brelse]
\funcproto void brelse(struct~buffer_head *bh);

This function frees the buffer \emph{bh}, previously allocated by
{\tt getblk}.

\item[bread]
\funcproto struct~buffer_head *bread(kdev_t dev, int block, int size);

This function allocates a buffer \emph{size} bytes in length, and fills
it with data from device \emph{dev}, starting at block number \emph{block}.

\item[block_write]
\funcproto int block_write(struct~inode *inode, struct~file *file,
				const char *buf, int count);

This function is the default implementation of file write.  It is used
by most of the Linux block drivers.  The function writes \emph{count}
bytes of data to the device specified by \emph{i_rdev} field of
\emph{inode}, starting at byte offset specified by \emph{f_pos} of
\emph{file}, from the buffer \emph{buf}.  The function returns 0 for
success, non-zero otherwise.

\item[block_read]
\funcproto int block_read(struct~inode *inode, struct~file *file,
				const char *buf, int count);

This function is the default implementation of file read.  It is used
by most of the Linux block drivers.  The function reads \emph{count}
bytes of data from the device specified by \emph{i_rdev} field of
\emph{inode}, starting at byte offset specified by \emph{f_pos} field of
\emph{file}, into the buffer \emph{buf}.  The function returns 0 for
success, non-zero otherwise.

\item[check_disk_change]
\funcproto int check_disk_change(kdev_t dev);

This function checks if media has been removed or changed in a
removable medium device specified by \emph{dev}.  It does so by invoking the
\emph{check_media_change} function in the driver's file operations
vector.  If a change has occurred, it calls the driver's \emph{revalidate}
function to validate the new media.  The function returns 0 if no
medium change has occurred, non-zero otherwise.

\item[request_dma]
\funcproto int request_dma(unsigned drq, const char *name);

This function allocates the DMA request line \emph{drq} for the calling
driver.  It returns 0 on success, non-zero otherwise.

\item[free_dma]
\funcproto void free_dma(unsigned drq);

This function frees the DMA request line \emph{drq} previously
allocated by {\tt request_dma}.

\item[disable_irq]
\funcproto void disable_irq(unsigned irq);

This function masks the interrupt request line \emph{irq} at the
interrupt controller.

\item[enable_irq]
\funcproto void enable_irq(unsigned irq);

This function unmasks the interrupt request line \emph{irq} at the
interrupt controller.

\item[request_irq]
\funcproto int request_irq(unsigned~int irq,
			   void (*handler)(int, struct),
			   unsigned long flags,
			   const char *device);

This function allocates the interrupt request line \emph{irq}, and
attach the interrupt handler \emph{handler} to it.  It returns 0 on
success, non-zero otherwise.

\item[free_irq]
\funcproto void free_irq(unsigned int irq);

This function frees the interrupt request line \emph{irq}, previously
allocated by {\tt request_irq}.

\item[kmalloc]
\funcproto void *kmalloc(unsigned~int size, int priority);

This function allocates \emph{size} bytes memory.
The \emph{priority} argument is a set of bitfields defined as follows:
\begin{icsymlist}
\item[GFP_BUFFER]	Not used by the drivers.
\item[GFP_ATOMIC]	Caller cannot sleep.
\item[GFP_USER]		Not used by the drivers.
\item[GFP_KERNEL]	Memory must be physically contiguous.
\item[GFP_NOBUFFER]	Not used by the drivers.
\item[GFP_NFS] 		Not used by the drivers.
\item[GFP_DMA]		Memory must be usable by the DMA controller.
			This means, on the x86, it must be below 16 MB, and
			it must not cross a 64K boundary.  This flag implies
			GFP_KERNEL.
\end{icsymlist}

\item[kfree]
\funcproto void kfree(void *p);

This function frees the memory \emph{p} previously allocated by
{\tt kmalloc}.

\item[vmalloc]
\funcproto void *vmalloc(unsigned long size);

This function allocates \emph{size} bytes of memory in kernel
virtual space that need not have underlying contiguous physical memory.

\item[check_region]
\funcproto int check_region(unsigned port, unsigned size);

Check if the I/O address space region starting at \emph{port} and
\emph{size} bytes in length, is available for use.  Returns 0 if region is
free, non-zero otherwise.

\item[request_region]
\funcproto void request_region(unsigned port, unsigned size,
				const char *name);

Allocate the I/O address space region starting at \emph{port} and
\emph{size} bytes in length.  It is the caller's responsibility to
make sure the region is free by calling {\tt check_region}, prior
to calling this routine.

\item[release_region]
\funcproto void release_region(unsigned port, unsigned size);

Free the I/O address space region starting at \emph{port} and
\emph{size} bytes in length, previously allocated by {\tt request_region}.

\item[add_wait_queue]
\funcproto void add_wait_queue(struct~wait_queue **q, struct~wait_queue *wait);

Add the wait element \emph{wait} to the wait queue \emph{q}.

\item[remove_wait_queue]
\funcproto void remove_wait_queue(struct~wait_queue **q,
				  struct~wait_queue *wait);

Remove the wait element \emph{wait} from the wait queue \emph{q}.

\item[down]
\funcproto void down(struct~semaphore *sem);

Perform a down operation on the semaphore \emph{sem}.  The caller blocks if
the value of the semaphore is less than or equal to 0.

\item[sleep_on]
\funcproto void sleep_on(struct~wait_queue **q, int interruptible);

Add the caller to the wait queue \emph{q}, and block it.
If \emph{interruptible} flag is non-zero, the caller can be woken up from
its sleep by a signal.

\item[wake_up]
\funcproto void wake_up(struct~wait_queue **q);

Wake up anyone waiting on the wait queue \emph{q}.

\item[wait_on_buffer]
\funcproto void wait_on_buffer(struct~buffer_head *bh);

Put the caller to sleep, waiting on the buffer \emph{bh}.
Called by drivers to wait for I/O completion on the buffer.

\item[schedule]
\funcproto void schedule(void);

Call the scheduler to pick the next task to run.

\item[add_timer]
\funcproto void add_timer(struct~timer_list *timer);

Schedule a time out.  The length of the time out and function to be called
on timer expiry are specified in \emph{timer}.

\item[del_timer]
\funcproto int del_timer(struct~timer_list *timer);

Cancel the time out \emph{timer}.
\end{icsymlist}


\subsection{Directory Structure}

The {\tt linux} subdirectory in the \oskit\ source tree
is organized as follows.
The top-level {\tt linux/dev} directory
contains all the glue code implemented by the Flux project
to squash the Linux drivers into the \oskit{} driver framework.
{\tt linux/fs} contains our glue for Linux filesystems,
and {\tt linux/shared} contains glue used by both components.
In general, everything \emph{except} the code in the {\tt linux/src} directory
was written by us, whereas everything under {\tt linux/src}
comes verbatim from Linux.
Each of the subdirectories of {\tt linux/src}
corresponds to the identically named subdirectories
of in the Linux kernel source tree.

Of course, there are a few necessary deviations from this rule:
a few of the Linux header and source files are slightly modified,
and a few of the Linux header files (but no source files)
were completely replaced.
The header files that were heavily modified include:

\begin{csymlist}
\item[linux/src/include/linux/sched.h]	Linux task and scheduling declarations
\end{csymlist}


\section{Block device drivers}

\begin{table}
\begin{tabular}[c]{|l|l||c|c|c|c|c|}
\hline
Name	& Description			& $V=P$
					  & {\tt jiffies}
					    & {\tt P+Y}
					      & {\tt current}
					        &	\\
\hline
\hline
cmd640.c	& CMD640 IDE Chipset	& & & & &	\\\hline
floppy		& Floppy drive		&*&*&*&*&	\\\hline
ide-cd.c	& IDE CDROM		& &*&*&*&	\\\hline
ide.c		& IDE Disk		& & & & &	\\\hline
rz1000.c	& RZ1000 IDE Chipset	& & & &	&	\\\hline
sd.c		& SCSI disk		& &*& &	&	\\\hline
sr.c		& SCSI CDROM		& & & &	&	\\\hline
triton.c	& Triton IDE Chipset	&*& & &	&	\\\hline
\end{tabular}
\label{linux-block-drivers}
\caption{Linux block device drivers}
\end{table}

\section{Network drivers}

Things drivers may want to do
that make emulation difficult:
\begin{itemize}
\item	Call the 16-bit BIOS.
\item	Use the system DMA controller.
\item	Assume kernel virtual addresses are equivalent to physical addresses.
\item	Assume kernel virtual addresses can be mapped to physical addresses
	merely by adding a constant offset.
\item	Implement timeouts by busy-waiting on a global clock-tick counter.
\item	Busy-wait for interrupts.
	XXX This means that the OS \emph{must} allow interrupts
	during execution of process-level driver code,
	and not just when all process-level activity is blocked.
\end{itemize}

\begin{table}
\begin{tabular}[c]{|l|l||c|c|c|c|c|}
\hline
Name	& Description			& $V=P$
					  & {\tt jiffies}
					    & {\tt P+Y}
					      & {\tt current}
					        &	\\
\hline
\hline
3c501.c		& 3Com 3c501 ethernet		& &*& &	&	\\\hline
3c503.c		& NS8390 ethernet		& &*& &	&	\\\hline
3c505.c		& 3Com Etherlink Plus (3C505)	& &*& &	&	\\\hline
3c507.c		& 3Com EtherLink16		& &*& &	&	\\\hline
3c509.c		& 3c509 EtherLink3 ethernet	& &*& &	&	\\\hline
3c59x.c		& 3Com 3c590/3c595 "Vortex"	& &*& &	&	\\\hline
ac3200.c	& Ansel Comm. EISA ethernet	& &*& &	&	\\\hline
apricot.c	& Apricot			&*&*& &	&	\\\hline
at1700.c	& Allied Telesis AT1700		& &*& &	&	\\\hline
atp.c		& Attached (pocket) ethernet	& &*& &	&	\\\hline
de4x5.c		& DEC DE425/434/435/500		& &*& &	&	\\\hline
de600.c		& D-link DE-600			& &*& &	&	\\\hline
de620.c		& D-link DE-620			& &*& &	&	\\\hline
depca.c		& DEC DEPCA \& EtherWORKS	& &*& &	&	\\\hline
e2100.c		& Cabletron E2100		& &*& &	&	\\\hline
eepro.c		& Intel EtherExpress Pro/10	& &*& &	&	\\\hline
eexpress.c	& Intel EtherExpress		& &*& &	&	\\\hline
eth16i.c	& ICL EtherTeak 16i \& 32	& &*& &	&	\\\hline
ewrk3.c		& DEC EtherWORKS 3		& &*& &	&	\\\hline
hp-plus.c	& HP PCLAN/plus			& &*& &	&	\\\hline
hp.c		& HP LAN			& &*& &	&	\\\hline
hp100.c		& HP10/100VG ANY LAN		& &*& &	&	\\\hline
lance.c		& AMD LANCE			&*&*& &	&	\\\hline
ne.c		& Novell NE2000			& &*& &	&	\\\hline
ni52.c		& NI5210 (i82586 chip)		& &*& &	&	\\\hline
ni65.c		& NI6510 (am7990 `lance' chip)	&*&*& &	&	\\\hline
seeq8005.c	& SEEQ 8005			& &*& &	&	\\\hline
sk_g16.c	& Schneider \& Koch G16		& &*& &	&	\\\hline
smc-ultra.c	& SMC Ultra			& &*& &	&	\\\hline
tulip.c		& DEC 21040			&*&*& &	&	\\\hline
wavelan.c	& AT\&T GIS (NCR) WaveLAN	& &*& &	&	\\\hline
wd.c		& Western Digital WD80x3	& &*& &	&	\\\hline
znet.c		& Zenith Z-Note			& &*& &	&	\\\hline
\end{tabular}
\label{linux-net-drivers}
\caption{Linux network drivers}
\end{table}

\section{SCSI drivers}

The Linux SCSI driver set includes
both the low-level SCSI host adapter drivers
and the high-level SCSI drivers for generic SCSI disks, tapes, etc.


\begin{table}
{\small
\begin{tabular}[c]{|l|l||c|c|c|c|c|}
\hline
Name	& Description			& $V=P$
					  & {\tt jiffies}
					    & {\tt P+Y}
					      & {\tt current}
					 	&	\\
\hline
\hline
53c7,8xx.c	& NCR 53C7x0, 53C8x0		&*&*& & &	\\\hline
AM53C974.c	& AM53/79C974 (PCscsi)		&*& & &	&	\\\hline
BusLogic.c	& BusLogic MultiMaster adapters	&*&*& &	&	\\\hline
NCR53c406a.c	& NCR53c406a			&*&*& & &	\\\hline
advansys.c	& AdvanSys SCSI Adapters	&*&*& &	&	\\\hline
aha152x.c	& Adaptec AHA-152x		& &*& &	&	\\\hline
aha1542.c	& Adaptec AHA-1542		&*&*& &	&	\\\hline
aha1740.c	& Adaptec AHA-1740		&*& & &	&	\\\hline
aic7xxx.c	& Adaptec AIC7xxx		&*&*& &	&	\\\hline
eata.c		& EATA 2.x DMA host adapters	& &*& &	&	\\\hline
eata_dma.c	& EATA/DMA host adapters	&*&*& &	&	\\\hline
eata_pio.c	& EATA/PIO host adapters	& &*& &	&	\\\hline
fdomain.c	& Future Domain TMC-16x0	& &*& &	&	\\\hline
in2000.c	& Always IN 2000		& &*& &	&	\\\hline
NCR5380.c	& Generic NCR5380		&*&*&*&	&	\\\hline
pas16.c		& Pro Audio Spectrum/Studio 16	& & & &	&	\\\hline
qlogic.c	& Qlogic FAS408			& &*& &	&	\\\hline
scsi.c		& SCSI middle layer		&*&*&*&*&	\\\hline
scsi_debug.c	& SCSI debug layer		& &*& &	&	\\\hline
seagate.c	& ST01,ST02, TMC-885		& &*& &	&	\\\hline
t128.c		& Trantor T128/128F/228		& & & &	&	\\\hline
u14-34f.c	& UltraStor 14F/34F		&*&*& &	&	\\\hline
ultrastor.c	& UltraStor 14F/24F/34F		&*& & &	&	\\\hline
wd7000.c	& WD-7000			&*&*& &	&	\\\hline
\end{tabular}
\label{linux-scsi-drivers}
\caption{Linux SCSI drivers}
}
\end{table}
